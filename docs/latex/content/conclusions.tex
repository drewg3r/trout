\conclusions


У роботі розглянуто основні підходи до розпізнавання емоційного стану людини за
статичним фронтальним зображенням її обличчя: штучні нейронні мережі, приховані
марківські моделі, порівняння з шаблоном та адаптивне нейронечітке виведення. У
результаті проведеного порівняльного аналізу за наперед визначеними критеріями
для вирішення поставленої задачі обрано модифікацію архітектури системи
адаптивного нейронечіткого виведення --- MANFIS.

Розглянуто й модифіковано методологію виділення вектора ознак із зображення
обличчя. Для кожної з ознак уведено множину нечітких значень, яких вона може
набувати. На множині цих нечітких значень побудовано правила продукції для
ідентифікації кожної емоції.

Спроектоване математичне забезпечення реалізовано програмно. Розроблену
програмну систему навчено на вибірці зі 180 зображень гібридним алгоритмом із
використанням перехресної перевірки для зупинки тренування.

У ході тестування виявлено показники ефективності: у середньому за всіма
емоціями показник ефективності навчання склав 85\%, а показник ефективності
узагальнення 73\%.

Основні положення дипломної роботи опубліковано у вигляді тез доповіді на
Міжнародній науково-технічній конференції SAIT 2016.
