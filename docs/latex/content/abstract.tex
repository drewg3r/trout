\abstractUkr


Дипломну роботу виконано на 28 аркушах, вона містить 2 додатки та
перелік посилань на використані джерела з 16 найменувань. У роботі наведено 4
рисунки та 2 таблиці.

Метою даної дипломної роботи є створення математичного та програмного
забезпечення для розпізнавання базових людських емоцій за статичним фронтальним
зображенням її обличчя.

У роботі проведено аналіз існуючих рішень указаної задачі — штучних нейронних
мереж, систем адаптивного нейронечіткого виведення та прихованих марківських
моделей. Виконано їх порівняння з погляду точності отримуваних розв’язків,
ефективності алгоритмів та пристосованості методів до використання нечітких
даних. Для розв’язання задачі в роботі вибрано метод адаптивного нейронечіткого
виведення.

Для кожної розглянутої емоції сформовано нечіткі продукційні правила.
Розроблено автоматизовану систему, що реалізує обраний метод. Виконано
тестування розробленої системи.

Основні положення дипломної роботи опубліковано у вигляді тез доповіді на
Міжнародній науково-технічній конференції SAIT 2016.

Ключові слова: емоція, система адаптивного нейронечіткого виведення, гібридний
алгоритм навчання, перехресна перевірка, продукційні правила, вектор ознак.




\abstractEng


The thesis is presented in 28 pages. It contains 2 appendixes and
bibliography of 16 references. Four figures and 2 tables are given in the
thesis. 

The goal of the thesis is to develop mathematical and software tools for
solving the problem of basic human emotion recognition by a static frontal
image of her face.

In the thesis, existing solutions are analyzed, such as artificial neural
networks, adaptive neuro-fuzzy inference systems, and hidden Markov models.
They are compared in terms of the accuracy of obtained results, algorithm
efficiency and method adaptation to fuzzy data. In the thesis, adaptive
neuro-fuzzy inference approach is used to solve the task.

Fuzzy production rules are formulated for each discussed emotion. The automated
system implementing the chosen method is developed. The developed system is
tested.

Main ideas of the thesis were published in the Proceedings of the International
Scientific and Technical Conference SAIT 2016.

Keywords: emotion, adaptive neuro-fuzzy inference system, hybrid learning
algorithm, cross-validation, production rules, feature vector.
