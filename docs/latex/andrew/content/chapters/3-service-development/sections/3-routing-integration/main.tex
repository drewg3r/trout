\section{Інтеграція веб-застосунку з data-engineering частиною}
\label{sec:integration-design}

У проекті є дві основні частини: функціонал пошуку маршрутів і веб-застосунок. Частина пошуку маршрутів зосереджена на наданні користувачам ефективних і точних рішень для прокладання маршрутів. А дана часитина проекту зосереджена на веб-застосунку, обробці запитів користувачів та наданні їм знайдених маршрутів.

В рамках частини пошуку маршрутів були розроблені та інтегровані різні компоненти. Сюди входить реалізація алгоритмів пошуку найкоротших маршрутів з урахуванням таких факторів, як розклад транспорту та обмеження. Крім того, були розроблені структури даних і моделі для зберігання і пошуку маршрутів, включаючи зупинки, маршрути і сполучення.

Для забезпечення безперешкодної інтеграції та підтримки було використано можливості фреймворку Django. У проекті Django інтеграція різних частин обертається навколо концепції app, або додатків. Додатки - це автономні модулі, які інкапсулюють певну функціональність і можуть бути легко підключені до проекту. Такий модульний підхід забезпечує кращу організацію, повторне використання коду та легку співпрацю між розробниками. Функція пошуку маршрутів була розроблена як окремий додаток Django, відмінний від інших частин проекту. Такий модульний дизайн сприяє організації коду, розподілу завдань і простоті  та інтеграції.

Завдяки використанню модулів Django, функція пошуку маршрутів може бути легко інтегрована з рештою проекту. Додатки, розроблені для інших аспектів проекту, можуть взаємодіяти з додатком пошуку маршрутів, використовуючи його можливості для отримання маршрутної інформації та включення її у свої функціональні можливості.

Загалом, архітектура проекту передбачає чітке розмежування між частиною пошуку маршрутів та іншими компонентами. Такий підхід уможливлює ефективну розробку, тестування та обслуговування, забезпечуючи надійність та інтегрованість системи.

