\section{Інтеграція веб-застосунку з data-engineering частиною}
\label{sec:integration-design}

У проекті Django інтеграція різних частин обертається навколо концепції app, або додатків. Додатки - це автономні модулі, які інкапсулюють певну функціональність і можуть бути легко підключені до проекту. Такий модульний підхід забезпечує кращу організацію, повторне використання коду та легку співпрацю між розробниками.

Щоб створити додаток Django, можна скористатися інтерфейсом командного рядка Django для створення необхідної файлової структури та шаблонного коду. Сюди входить каталог програми, конфігураційні файли та початкові налаштування. Кожен додаток зазвичай складається з моделей, представлень, шаблонів та інших компонентів, необхідних для реалізації його специфічної функціональності.

У випадку сервісу пошуку маршрутів, проект матиме спеціальний додаток Django, відповідальний за роботу з функціоналом пошуку маршрутів. Цей додаток міститиме моделі, що представляють дані про маршрут а також функції, що забезпечують пошук маршруту.

Потужна система маршрутизації URL-адрес Django відіграє вирішальну роль в інтеграції програми пошуку маршрутів з рештою проекту. URL-адреси зіставляються з конкретними запитами, що дозволяє користувачам отримувати доступ до різних частин програми за різними URL-адресами. Визначивши відповідні шаблони URL-адрес і пов'язавши їх з відповідними поданнями, ви можете гарантувати, що запити на пошук маршрутів будуть перенаправлятися до відповідних подань і оброблятися відповідним чином.

Крім того, Django надає ряд можливостей, які полегшують інтеграцію різних частин проекту. Система шаблонів дозволяє повторно використовувати спільні елементи та рендерити динамічний контент, забезпечуючи послідовну та гнучку презентацію у всьому додатку. ORM (об'єктно-реляційне відображення) Django спрощує операції з базами даних, дозволяючи вам безперешкодно взаємодіяти з базовою базою даних.

Використання модульного підходу Django, разом з його надійною маршрутизацією URL, системою шаблонів, ORM та механізмами автентифікації, полегшує плавну інтеграцію додатку пошуку маршрутів з іншими частинами проекту. Це сприяє повторному використанню коду, підтримці та масштабованості, дозволяючи розробникам працювати над різними компонентами незалежно та безперешкодно інтегрувати їх для створення цілісного та функціонального веб-додатку.