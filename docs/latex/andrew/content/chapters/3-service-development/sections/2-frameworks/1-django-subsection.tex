\subsection{Django}
\label{subsec:django-subsection}

Django - це високорівневий веб-фреймворк, написаний на мові Python, який відповідає архітектурному шаблону "модель-вигляд-контролер" (MVC). Він покликаний спростити та прискорити процес створення веб-додатків, надаючи надійний набір інструментів та функцій.

Однією з ключових переваг Django є його потужна система об'єктно-реляційного відображення (ORM). Вона дозволяє розробникам взаємодіяти з базою даних за допомогою об'єктів Python, усуваючи необхідність писати складні SQL-запити. ORM підтримує різні бекенди баз даних, включаючи PostgreSQL, MySQL, SQLite і Oracle, забезпечуючи гнучкість і сумісність.

Django також має вбудовану систему автентифікації, що дозволяє легко керувати реєстрацією користувачів, логінами та паролями. Вона забезпечує безпечну і настроювану систему автентифікації та авторизації користувачів, допомагаючи розробникам впроваджувати контроль доступу і специфічні для користувача функції у своїх додатках.

Ще однією визначною особливістю Django є його механізм шаблонів, який дозволяє відокремлювати HTML-код від коду Python. Це дозволяє розробникам створювати динамічні веб-сторінки, вбудовуючи код Python у шаблони HTML, що спрощує створення динамічного контенту та обробку даних.

Django дотримується принципу "Не повторюй себе" (DRY) і сприяє багаторазовому використанню коду завдяки своєму модульному дизайну. Він пропонує широкий спектр готових компонентів, відомих як "додатки Django", які надають такі функціональні можливості, як інтерфейс адміністратора, обробка форм, кешування та інше. Ця розгалужена екосистема багаторазових додатків економить час і зусилля розробників, дозволяючи їм зосередитися на унікальних аспектах своїх додатків.

Крім того, Django підкреслює важливість безпеки. Він включає в себе численні функції безпеки, включаючи захист від поширених веб-уразливостей, таких як міжсайтовий скриптинг (XSS) і підробка міжсайтових запитів (CSRF). Акцент Django на безпеці допомагає розробникам створювати надійні та безпечні веб-додатки.

Крім того, Django має активну спільноту, яка надає велику документацію, навчальні посібники та пакети сторонніх розробників. Спільнота активно підтримує та оновлює фреймворк, забезпечуючи його стабільність, надійність та сумісність з новими версіями Python та веб-стандартами.

Загалом, Django - це комплексний веб-фреймворк, який спрощує розробку веб-додатків, пропонуючи потужні функції, ефективний ORM, вбудовану автентифікацію, движок шаблонів та безліч компонентів для багаторазового використання. Зосередженість на безпеці, дотримання найкращих практик та процвітаюча спільнота роблять його популярним вибором для створення масштабованих та багатофункціональних веб-додатків.


Переваги Django:
\begin{itemize}
\item Комплексний фреймворк з багатим набором функцій та інструментів
\item Потужне об'єктно-реляційне відображення (ORM) для взаємодії з базами даних
\item Вбудована система автентифікації для управління користувачами
\item Движок шаблонів для динамічного створення контенту
\item Модульний дизайн, що сприяє повторному використанню коду
\item Наголос на безпеку з вбудованим захистом від поширених вразливостей
\item Широка документація та потужна підтримка спільноти
\end{itemize}

Недоліки Django:
\begin{itemize}
\item Обмежена гнучкість у виборі компонентів
\item Накладні витрати на продуктивність у певних сценаріях через надійність та абстракції фреймворку
\item Вимагає дотримання угод Django, які можуть обмежувати певні налаштування або відхилятися від особистих уподобань
\end{itemize}