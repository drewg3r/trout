\subsection{Отримання, обробка та відображення даних про транспорт}
\label{subsec:route-management-subsection}

В основі сервісу пошуку маршрутів лежить ефективна обробка та використання транспортних даних. У цьому розділі буде заглиблено у складний процес отримання, обробки та відображення транспортних даних, який є основою функціональності сервісу для пошуку маршрутів. 

Для надання точної та актуальної транспортної інформації в сервісі використовується інтерактивний користувацький інтерфейс, за допомогою якого адміністратор чи представник транспортної компанії може вводити дані про транспортні маршрути, щоб колристувачі могли ними користуватись. Також інтерфейс дозволяє користувачам шукати маршрути, що підходять їх потребам, вказавши бажаний час відправлення, початкову зупинку та пункт призначення.

Після отримання даних з інтерфейсу система для пошуку маршрутів використовує складні алгоритми і методи обробки даних для перетворення введених користувачем даних у придатний для використання формат. Це включає перевірку введених даних, зіставлення їх з наявними транспортними даними та визначення найкращих можливих маршрутів на основі вподобань користувача.

Оброблені дані потім організовуються в комплексну і взаємопов'язану транспортну мережу, де встановлюються зв'язки між різними маршрутами, зупинками і пунктами призначення. Це дозволяє сервісу генерувати точні та ефективні маршрути на основі даних та вподобань користувачів. Транспортні дані доповнюються додатковою контекстною інформацією, такою як дані про об'єкти, орієнтири та відповідні геопросторові дані, що збагачує загальний користувацький досвід.

Відображення оброблених транспортних даних у зручний та візуально привабливий спосіб має вирішальне значення для забезпечення безперешкодної роботи користувачів. В даному сервісі використовуються сучасні методи візуалізації даних та інтерактивні карти для представлення зупинок. Інтерфейс розроблений таким чином, щоб бути інтуїтивно зрозумілим і зручним для навігації, що дозволяє користувачам легко досліджувати та взаємодіяти з транспортними даними.

Зображення сторінки для додавання даних

Зораження сторінки для відображення доданих маршрутів


Щоб зробити можливим додавання транспортних маршрутів до системи пошуку маршрутів, потрібно реалізували функціонал, який дозволяє авторизованим адміністраторам чи відповідальним особам додавати нові маршрути. Використовуючи архітектуру Django Model-View-Controller (MVC), потрібно створити спеціальну функцію для відображення транспортних маршрутів та форму для створення транспортного маршруту. Форма збирає необхідну інформацію для створення маршруту. Після відправки даних, вони перевіряються спеціальною функцією і створюється новий об'єкт маршруту в базі даних, використовуючи ORM Django. Це гарантує, що доданий маршрут зберігається постійно і до нього можна отримати доступ для подальшої обробки та відображення.

Обробка транспортних маршрутів включає в себе маніпуляції та аналіз даних маршруту для отримання змістовної інформації. Для цьог використовуються потужні можливості запитів Django для отримання відповідних маршрутів з бази даних на основі певних критеріїв. Наприклад, можна отримати маршрути, які задовольняють певним параметрам, таким як відстань або час. Маючи отримані дані про маршрут, можна виконувати розрахунки і застосовувати алгоритми для визначення таких факторів, як найкоротший маршрут, оптимальний розклад або альтернативні маршрути. Ці оброблені результати можна зберігати або використовувати для подальших операцій, таких як рекомендації щодо маршруту або оптимізація розкладу.

Щоб надати користувачам інтуїтивно зрозуміле та інформативне відображення транспортних маршрутів, використовується система шаблонів Django та HTML/CSS для створення динамічних і візуально привабливих веб-сторінок. За допомогою цього розробляються шаблони, які ефективно представляють деталі маршруту, включаючи пункт відправлення, пункт призначення, проміжні зупинки, пересадки, час у дорозі та види транспорту.

Реалізувавши ці функції в рамках проекту на Python та Django, можна легко обробляти додавання, обробку та відображення транспортних маршрутів. Це забезпечує ефективне управління даними, точні розрахунки маршрутів і візуально привабливий користувальницький інтерфейс, що дозволяє користувачам приймати обґрунтовані рішення щодо подорожей на основі доступних варіантів транспорту.