\subsection{Відображення знайденого маршруту}
\label{subsec:route-displaying-subsection}

Сторінка для відображення знайденого маршруту є важливим компонентом сервысу для пошуку маршрутів, надаючи користувачам комплексне уявлення про підібрані для них маршрути. У цьому розділі розглядаються функції та можливості сторінки для відображення знайденого маршруту.

Основна функція сторінки для відображення знайденого маршруту - надати користувачам візуально привабливе та інформативне представлення підібраних для них маршрутів. Після того, як користувачі здійснили пошук маршруту, на цій сторінці відображається знайдений маршрут разом з різними деталями та інтерактивними елементами. Користувачі можуть переглядати покрокові вказівки, карти з зупинками, пересадки, приблизний час у дорозі та іншу інформацію, яка допомагає їм ефективно орієнтуватися під час подорожі.

Однією з ключових причин, чому користувачі покладаються на сторінку відображення маршруту, є її здатність забезпечити чітке розуміння всього маршруту від початку до кінця. Вона представляє користувачам огляд прокладеного маршруту, виділяючи послідовність маршрутних точок, об'єктів і будь-яких необхідних змін режиму руху. Таке комплексне відображення дає змогу користувачам підготуватися до подорожі, забезпечуючи цілісне розуміння майбутнього маршруту та передбачаючи будь-які потенційні виклики або орієнтири на шляху.

Крім того, сторінка відображення маршруту пропонує користувачам низку інтерактивних функцій для покращення вивчення маршруту. Користувачі можуть збільшувати і зменшувати масштаб мапи і досліджувати конкретні об'єкти або орієнтири вздовж маршруту. Ці інтерактивні елементи надають користувачам динамічний і цікавий досвід, дозволяючи їм ознайомитися з навколишнім середовищем, визначити орієнтири і приймати обґрунтовані рішення під час подорожі.

Крім того, на сторінці відображення маршруту пріоритетом є доступність та зручність для користувачів. Вона розроблена таким чином, щоб бути адаптивною та зрозумілою, забезпечуючи безперебійний перегляд на різних пристроях та екранах різного розміру. Це дозволяє користувачам отримувати доступ до обраних маршрутів в такому вигляді, в якому їм зручно, і взаємодіяти з ними, незалежно від того, чи використовують вони стаціонарний комп'ютер, планшет, мобільний пристрій чи будь-який інший пристрій, де є браузер.

Таким чином, сторінка для відображення знайденого маршруту відіграє вирішальну роль у наданні користувачам комплексного та інтерактивного відображення обраних ними маршрутів. Пропонуючи детальні вказівки, інтерактивні карти, вона дозволяє користувачам впевнено та ефективно орієнтуватися у транспортних мережах під час подорожі, чи на етапі її планування. Зручний інтерфейс, цікаві інтерактивні елементи та доступність роблять сторінку для відображення знайденого маршруту цінним ресурсом для тих, хто шукає зручне та інформативне відображення маршрутів.


Зображення сторінки для пошуку маршрутів


Щоб реалізувати функцію відображення маршруту у нашому веб-додатку, ми використовуємо потужні можливості та гнучкість фреймворку Django. В основі реалізації лежить використання представлень і шаблонів Django. Ми створюємо спеціальне представлення, яке отримує обрану інформацію про маршрут з бекенду і відображає її за допомогою відповідного шаблону. Представлення отримує необхідні дані, такі як покрокові інструкції, точки маршруту, орієнтовний час у дорозі та інші відповідні деталі, з нашої внутрішньої бази даних або зовнішніх API.

Система шаблонів Django дозволяє нам структурувати і форматувати інформацію про маршрут у візуально привабливий спосіб. За допомогою використання HTML, CSS та мови шаблонів Django можна створити зручний та функціональний інтерфейс, який представляє деталі маршруту в чіткій, зручній, зрозумілій та організованій формі. Шаблон може включє інтерактивні елементи, такі як карти, функцію масштабування та перемикання між різними видами карт, щоб покращити користувацький досвід.

Крім того, система автентифікації та авторизації користувачів Django дозволяє представникам транспортної компанії керувати доступом до управляння транспортними даними і надавати доступ до цієї функції лише відповідальним особам. Це дозволяє відділити управління транспортом від функцій доступних звичайним користувачам, таким як перегляд маршрутів, пошук маршрутів і інших.

Крім того, підтримка Django адаптивного веб-дизайну та мобільної оптимізації гарантує, що сторінка з маршрутом буде доступною і функціональною на різних пристроях і з різними розмірами екрану. Використовуючи адаптивні CSS фреймворки, такі як Bootstrap, ми можемо досягти адаптивного і візуально привабливого інтерфейсу користувача, який адаптується до пристрою користувача, будь то настільний комп'ютер, планшет або мобільний телефон.