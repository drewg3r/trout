\subsection{Python}
\label{subsec:python-subsection}

Python - це універсальна і широко поширена мова програмування, відома своєю простотою, читабельністю та розгалуженою екосистемою. Це інтерпретована мова високого рівня, яка робить акцент на читабельності та виразності коду, що робить її чудовим вибором для розробки веб-додатків з можливостями пошуку маршрутів.

Мова програмування Python пропонує багату колекцію бібліотек і фреймворків, які спрощують процеси веб-розробки. Flask і Django - два популярні фреймворки, які забезпечують надійну основу для створення веб-додатків на Python. Flask - це легкий і гнучкий фреймворк, який фокусується на простоті і зручності використання, що робить його ідеальним для малих і середніх проектів. Django, з іншого боку, є надійним і всеосяжним фреймворком, який слідує архітектурному шаблону Model-View-Controller (MVC), пропонуючи такі функції, як маршрутизація URL-адрес, ORM (об'єктно-реляційне відображення) баз даних і автентифікація користувачів "з коробки".

Окрім фреймворків, мова програмування Python може похвалитися розгалуженою екосистемою бібліотек, які є безцінними для додатків для пошуку маршрутів. Наприклад, бібліотека NetworkX надає структури графових даних та алгоритми, які необхідні для реалізації алгоритмів маршрутизації. Бібліотека GeoPandas дозволяє маніпулювати просторовими даними та аналізувати їх, забезпечуючи інтеграцію з джерелами географічних даних для точного планування маршрутів. Крім того, такі популярні бібліотеки, як NumPy та SciPy, надають ефективні можливості чисельних обчислень, які можна використовувати в алгоритмах оптимізації маршрутів.

Універсальність мови програмування Python виходить за рамки веб-розробки, оскільки її також можна використовувати для написання сценаріїв, аналізу даних, машинного навчання тощо. Простота мови та широка підтримка спільноти роблять її привабливим вибором для розробників усіх рівнів кваліфікації. Крім того, мова програмування Python має широку екосистему сторонніх пакетів, доступних через Python Package Index (PyPI), що забезпечує доступ до широкого спектру інструментів і ресурсів для вдосконалення додатків для пошуку маршрутів.

Отже, Python - це потужна та універсальна мова програмування, яка пропонує широкий спектр інструментів, фреймворків та бібліотек для розробки веб-додатків з можливостями пошуку маршрутів. Її простота, зрозумілість та розгалужена екосистема роблять її популярною серед розробників, дозволяючи ефективно та результативно розробляти рішення для планування та оптимізації маршрутів.

Переваги Python:
\begin{itemize}
    \item Читабельність: мова програмування Python робить акцент на читабельності коду завдяки чистому і простому синтаксису, що робить його легким для розуміння і підтримки.
    \item Універсальність: Python - це універсальна мова, яку можна використовувати для веб-розробки, аналізу даних, наукових обчислень, штучного інтелекту тощо.
    \item Велика спільнота та екосистема: мова програмування Python має велику спільноту розробників, які роблять свій внесок у розгалужену екосистему бібліотек, фреймворків та інструментів, надаючи рішення для різних сфер та додатків.
    \item Крос-платформна сумісність: код на мові програмування Python може працювати на різних операційних системах, включаючи Windows, macOS і різні Unix-подібні системи, що забезпечує портативність і гнучкість.
    \item Швидкий розвиток: лаконічний синтаксис мови програмування Python та широка підтримка бібліотек сприяють швидкій розробці, дозволяючи розробникам створювати додатки швидко та ефективно.
    \item Можливості інтеграції: Python легко інтегрується з іншими мовами, такими як C/C++, Java та .NET, дозволяючи розробникам використовувати існуючий код та бібліотеки.
\end{itemize}

Недоліки Python:
\begin{itemize}
    \item Обмеження продуктивності: порівняно з мовами нижчого рівня, такими як C або C++, Python може мати обмеження продуктивності через свою інтерпретовану природу, що може вплинути на швидкість виконання обчислювально інтенсивних завдань.
    \item Споживання пам'яті: споживання пам'яті мова програмування Python може бути вищим у порівнянні з деякими іншими мовами, що може вплинути на масштабованість та ресурсоємність додатків.
    \item Не підходить для всіх випадків використання: хоча мова програмування Python є універсальною мовою, вона може бути не найкращим вибором для певних випадків використання, які вимагають низькорівневого контролю або суворої оптимізації продуктивності.
    \item Сумісність версій: мова програмування Python має декілька версій, що може призвести до проблем сумісності при інтеграції зі сторонніми бібліотеками або роботі з кодовими базами, розробленими у різних версіях Python.
\end{itemize}
