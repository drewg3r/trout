\section{Вибір мови програмування}
\label{sec:programming-language}

Для розробки веб-застосунку, який би вирішував поставлену задачу існує цілий ряд мов програмування, які пропонують відповідні можливості. Мова програмування Python, відома своєю простотою і зрозумілістю, надає надійні бібліотеки і фреймворки, такі як Flask і Django, які полегшують веб-розробку і безперешкодну інтеграцію алгоритмів пошуку маршрутів. Мова програмування JavaScript, будучи мовою де-факто для веб-рішень, пропонує фреймворки, такі як Node.js, і бібліотеки, такі як Leaflet.js і Mapbox.js, що робить її широко прийнятим вибором для створення інтерактивних застосунків. Мова програмування Java, з її акцентом на масштабованість і розгалужену екосистему, надає такі фреймворки, як Spring і JavaServer Faces (JSF), які дозволяють розробляти ефективні веб-додатки з інтегрованою функцією пошуку маршрутів. Крім того, такі мови програмування, як Ruby, Go і TypeScript, також пропонують достойні варіанти, кожен з яких має свої сильні сторони і фреймворки, що відповідають вимогам веб-розробки та пошуку маршрутів. Вибір мови програмування повинен визначатися такими факторами, як специфічні вимоги проекту, кваліфікація команди розробників та наявність відповідних бібліотек і фреймворків.


%Subsections
\subsection{Python}
\label{subsec:python-subsection}

Python - це універсальна і широко поширена мова програмування, відома своєю простотою, читабельністю та розгалуженою екосистемою. Це інтерпретована мова високого рівня, яка робить акцент на читабельності та виразності коду, що робить її чудовим вибором для розробки веб-додатків з можливостями пошуку маршрутів.

Мова програмування Python пропонує багату колекцію бібліотек і фреймворків, які спрощують процеси веб-розробки. Flask і Django - два популярні фреймворки, які забезпечують надійну основу для створення веб-додатків на Python. Flask - це легкий і гнучкий фреймворк, який фокусується на простоті і зручності використання, що робить його ідеальним для малих і середніх проектів. Django, з іншого боку, є надійним і всеосяжним фреймворком, який слідує архітектурному шаблону Model-View-Controller (MVC), пропонуючи такі функції, як маршрутизація URL-адрес, ORM (об'єктно-реляційне відображення) баз даних і автентифікація користувачів "з коробки".

Окрім фреймворків, мова програмування Python може похвалитися розгалуженою екосистемою бібліотек, які є безцінними для додатків для пошуку маршрутів. Наприклад, бібліотека NetworkX надає структури графових даних та алгоритми, які необхідні для реалізації алгоритмів маршрутизації. Бібліотека GeoPandas дозволяє маніпулювати просторовими даними та аналізувати їх, забезпечуючи інтеграцію з джерелами географічних даних для точного планування маршрутів. Крім того, такі популярні бібліотеки, як NumPy та SciPy, надають ефективні можливості чисельних обчислень, які можна використовувати в алгоритмах оптимізації маршрутів.

Універсальність мови програмування Python виходить за рамки веб-розробки, оскільки її також можна використовувати для написання сценаріїв, аналізу даних, машинного навчання тощо. Простота мови та широка підтримка спільноти роблять її привабливим вибором для розробників усіх рівнів кваліфікації. Крім того, мова програмування Python має широку екосистему сторонніх пакетів, доступних через Python Package Index (PyPI), що забезпечує доступ до широкого спектру інструментів і ресурсів для вдосконалення додатків для пошуку маршрутів.

Отже, Python - це потужна та універсальна мова програмування, яка пропонує широкий спектр інструментів, фреймворків та бібліотек для розробки веб-додатків з можливостями пошуку маршрутів. Її простота, зрозумілість та розгалужена екосистема роблять її популярною серед розробників, дозволяючи ефективно та результативно розробляти рішення для планування та оптимізації маршрутів.

Переваги Python:
\begin{itemize}
    \item Читабельність: мова програмування Python робить акцент на читабельності коду завдяки чистому і простому синтаксису, що робить його легким для розуміння і підтримки.
    \item Універсальність: Python - це універсальна мова, яку можна використовувати для веб-розробки, аналізу даних, наукових обчислень, штучного інтелекту тощо.
    \item Велика спільнота та екосистема: мова програмування Python має велику спільноту розробників, які роблять свій внесок у розгалужену екосистему бібліотек, фреймворків та інструментів, надаючи рішення для різних сфер та додатків.
    \item Крос-платформна сумісність: код на мові програмування Python може працювати на різних операційних системах, включаючи Windows, macOS і різні Unix-подібні системи, що забезпечує портативність і гнучкість.
    \item Швидкий розвиток: лаконічний синтаксис мови програмування Python та широка підтримка бібліотек сприяють швидкій розробці, дозволяючи розробникам створювати додатки швидко та ефективно.
    \item Можливості інтеграції: Python легко інтегрується з іншими мовами, такими як C/C++, Java та .NET, дозволяючи розробникам використовувати існуючий код та бібліотеки.
\end{itemize}

Недоліки Python:
\begin{itemize}
    \item Обмеження продуктивності: порівняно з мовами нижчого рівня, такими як C або C++, Python може мати обмеження продуктивності через свою інтерпретовану природу, що може вплинути на швидкість виконання обчислювально інтенсивних завдань.
    \item Споживання пам'яті: споживання пам'яті мова програмування Python може бути вищим у порівнянні з деякими іншими мовами, що може вплинути на масштабованість та ресурсоємність додатків.
    \item Не підходить для всіх випадків використання: хоча мова програмування Python є універсальною мовою, вона може бути не найкращим вибором для певних випадків використання, які вимагають низькорівневого контролю або суворої оптимізації продуктивності.
    \item Сумісність версій: мова програмування Python має декілька версій, що може призвести до проблем сумісності при інтеграції зі сторонніми бібліотеками або роботі з кодовими базами, розробленими у різних версіях Python.
\end{itemize}


\subsection{JavaScript}
\label{subsec:js-subsection}

JavaScript - це універсальна і широко використовувана мова програмування, яка в першу чергу використовується для розробки інтерактивних і динамічних веб-додатків. Вона підтримується всіма основними веб-браузерами, що робить її фундаментальною технологією для інтерфейсної веб-розробки. JavaScript - це високорівнева, інтерпретована мова з простим і гнучким синтаксисом, яка дозволяє розробникам створювати багатий і цікавий користувацький досвід в Інтернеті.

Однією з ключових переваг мови програмування JavaScript є її здатність працювати безпосередньо в браузері, що дозволяє створювати сценарії на стороні клієнта. Це означає, що код JavaScript може виконуватися на пристрої користувача, зменшуючи потребу в обробці на стороні сервера і підвищуючи швидкість відгуку веб-додатків. Мова програмування JavaScript може маніпулювати елементами HTML, обробляти користувацькі події, виконувати перевірку вводу та динамічно оновлювати вміст і зовнішній вигляд веб-сторінок.

Окрім розробки інтерфейсів, мова програмування JavaScript також використовується для розробки бекенд-версій з появою серверних фреймворків JavaScript, таких як Node.js. Node.js дозволяє розробникам створювати масштабовані та ефективні веб-сервери та додатки за допомогою мови програмування JavaScript. За допомогою Node.js JavaScript можна використовувати для обробки серверної логіки, виконання операцій з базами даних та створення RESTful API.

Мова програмування JavaScript має велику екосистему бібліотек та фреймворків, які розширюють його можливості та спрощують завдання веб-розробки. Такі популярні бібліотеки, як React, Angular та Vue.js, надають потужні інструменти для створення складних користувацьких інтерфейсів та управління станом у великомасштабних додатках. Ці фреймворки підвищують продуктивність завдяки багаторазовому використанню компонентів, ефективному зв'язуванню даних та широкій підтримці спільноти.

Більше того, мова програмування JavaScript вийшов за межі веб-розробки і тепер використовується для розробки мобільних додатків (React Native, Ionic), десктопних додатків (Electron) і навіть машинного навчання (TensorFlow.js). Така універсальність дозволяє розробникам використовувати свої навички JavaScript на різних платформах і в різних сферах.

Незважаючи на свої сильні сторони, мова програмування JavaScript має деякі обмеження. Він може бути схильний до проблем сумісності з браузерами, оскільки різні браузери можуть дещо по-різному інтерпретувати JavaScript-код. Крім того, гнучкість мови програмування JavaScript може призвести до потенційних пасток, якщо використовувати його необережно, наприклад, проблеми зі змінними в області видимості та примусове використання типів. Однак, правильне кодування та використання сучасних функцій та інструментів мови програмування JavaScript може допомогти пом'якшити ці проблеми.

Загалом, JavaScript - це потужна і широко розповсюджена мова програмування, яка дозволяє розробникам створювати динамічні та інтерактивні веб-додатки. Її універсальність, велика екосистема та широка підтримка браузерами роблять її популярним вибором для фронтенд- та бекенд-веб-розробки, а також для інших сфер застосування.

Переваги:
\begin{itemize}
    \item Широка сумісність з браузерами: JavaScript підтримується всіма основними веб-браузерами, що забезпечує широку сумісність веб-додатків.
    \item Інтерактивна взаємодія з користувачем: JavaScript дозволяє    створювати динамічні та інтерактивні користувацькі інтерфейси, покращуючи загальний користувацький досвід.
    \item Широка екосистема: JavaScript має широку екосистему бібліотек, фреймворків та інструментів, які спрощують завдання розробки та прискорюють реалізацію проектів.
    \item Багаторазове використання: бібліотеки та фреймворки JavaScript надають багаторазові компоненти та модулі, що полегшує повторне використання коду та пришвидшує розробку.
\end{itemize}

Недоліки:
\begin{itemize}
    \item Сумісність з браузерами: різні браузери можуть по-різному інтерпретувати код JavaScript, що вимагає додаткових зусиль для тестування та налагодження кросбраузерної сумісності.
    \item Обмежена продуктивність: інтерпретована природа JavaScript може призвести до зниження продуктивності порівняно з мовами нижчого рівня, особливо для завдань з інтенсивними обчисленнями.
    \item Проблеми безпеки: як сценарії на стороні клієнта, код JavaScript доступний користувачам, що робить його вразливим до таких атак, як впровадження коду та міжсайтовий скриптинг, якщо він не захищений належним чином.
    \item Відсутність строгої типізації: JavaScript має динамічну типізацію, що може призвести до потенційних помилок і ускладнити підтримку коду.
    \item Крива навчання: JavaScript має криву навчання, особливо для розробників, які переходять з інших мов, через його унікальні особливості та асинхронну модель програмування.
\end{itemize}

\subsection{Java}
\label{subsec:java-subsection}

Java - це широко використовувана мова програмування, відома своєю універсальністю, продуктивністю та надійністю. Вона особливо популярна у сфері корпоративної веб-розробки, де її об'єктно-орієнтована природа та розгалужена екосистема роблять її чудовим вибором для створення масштабованих та безпечних веб-додатків[8].

Основна перевага мови програмування Java полягає в її незалежності від платформи. Код Java компілюється у байт-код, який може працювати на будь-якій платформі за допомогою віртуальної машини Java (JVM). Це дозволяє розробникам писати код один раз і розгортати його на різних операційних системах, що робить Java ідеальним вибором для веб-додатків, які повинні бути сумісними з різними середовищами.

У контексті веб-розробки мова програмування Java зазвичай використовується для програмування на стороні сервера. Фреймворки на основі Java, такі як JavaServer Pages (JSP), JavaServer Faces (JSF) та Java Servlets, забезпечують основу для створення динамічних та інтерактивних веб-додатків. Ці фреймворки дозволяють розробникам обробляти HTTP-запити, генерувати динамічний веб-контент, керувати сеансами користувачів і взаємодіяти з базами даних.

Веб-додатки на Java часто використовують архітектуру Model-View-Controller (MVC), яка сприяє розділенню завдань і модульній розробці. Такі фреймворки, як Spring MVC та JavaServer Faces (JSF), забезпечують надійну реалізацію MVC, дозволяючи розробникам структурувати свої додатки та ефективно керувати складною бізнес-логікою.

Розгалужена екосистема Java також включає широкий спектр бібліотек та фреймворків, які покращують веб-розробку. Spring Framework, наприклад, надає комплексну підтримку для створення додатків корпоративного рівня, включаючи такі функції, як встановлення залежностей, доступ до даних та безпека. Інші фреймворки, такі як Hibernate, спрощують роботу з базами даних, а Apache Struts пропонує потужний фреймворк MVC для веб-додатків.

Крім того, мова програмування Java підтримує використання стандартної бібліотеки тегів JavaServer Pages (JSTL), яка надає набір тегів для поширених завдань веб-розробки, таких як ітерації, умови та форматування. Це полегшує розробникам створення динамічного контенту на веб-сторінках.

Крім того, мова програмування Java пропонує потужні функції безпеки та розвинений набір інструментів і бібліотек для захисту веб-додатків. Служба автентифікації та авторизації Java (JAAS) та фреймворки, такі як Spring Security, дозволяють розробникам впроваджувати автентифікацію, контроль доступу та інші заходи безпеки для захисту своїх веб-додатків.

Однак важливо зазначити, що веб-розробка на Java, як правило, передбачає більше налаштувань та конфігурацій порівняно з іншими мовами. Крива навчання може бути крутішою, особливо для початківців, через складність мови та різноманітність доступних фреймворків. Крім того, веб-додатки на Java можуть вимагати більше пам'яті та обчислювальної потужності порівняно з легшими альтернативами.

Загалом, надійність, незалежність від платформи та розгалужена екосистема мови програмування Java роблять її чудовим вибором для веб-розробки, особливо в корпоративному середовищі. Її продуктивність, функції безпеки та масштабованість роблять її добре придатною для створення великомасштабних, критично важливих веб-додатків.

Переваги Java:
\begin{itemize}
    \item Незалежність від платформи: код на Java працює на будь-якій платформі з віртуальною машиною Java (JVM), що забезпечує чудову портативність.
    \item Об'єктно-орієнтоване програмування: об'єктно-орієнтована природа Java дозволяє створювати модульний і багаторазовий код, що сприяє легкості супроводу і масштабованості.
    \item Надійність: сувора перевірка під час компіляції та обробка винятків у Java сприяють створенню стабільних та надійних додатків.
    \item Велика екосистема: Java має велику екосистему бібліотек, фреймворків та інструментів, які підтримують різні аспекти розробки, прискорюючи процес розробки.
    \item Безпека: Java надає вбудовані засоби безпеки, такі як пісочниця, перевірка байт-коду та криптографічні бібліотеки для розробки безпечних додатків.
\end{itemize}

Недоліки Java:
\begin{itemize}
    \item Крива навчання: Java має круту криву навчання, особливо для початківців, через складність синтаксису та необхідність розуміння об'єктно-орієнтованих принципів.
    \item Багатослівність: синтаксис Java може бути багатослівним, що вимагає більшої кількості рядків коду порівняно з іншими мовами для досягнення подібної функціональності.
    \item Споживання пам'яті: програми на Java зазвичай потребують більше пам'яті порівняно з легшими мовами, що може бути важливим фактором для середовищ з обмеженими ресурсами.
    \item Час запуску: програми на Java, як правило, мають довший час запуску порівняно з інтерпретованими мовами, оскільки вони потребують ініціалізації JVM та завантаження залежностей.
    \item Обмежений паралелізм: хоча Java забезпечує підтримку паралелізму за допомогою потоків і бібліотек, таких як java.util.concurrent, керування паралельним програмуванням може бути складним і схильним до помилок.
\end{itemize}

\subsection{Go}
\label{subsec:go-subsection}

Go - це статично типізована, скомпільована мова програмування, розроблена компанією Google. Вона створена для того, щоб бути ефективною, лаконічною та добре масштабованою. Хоча Go є мовою загального призначення, вона набула популярності у спільноті веб-розробників завдяки своїй чудовій підтримці для створення високопродуктивних веб-додатків[10].

Простота і мінімалізм мови програмування Go роблять її привабливим вибором для веб-розробки. Її синтаксис чистий і легкий для читання, що дозволяє розробникам писати лаконічний і виразний код. Мова програмування Go робить акцент на читабельності коду, що полегшує його розуміння та підтримку великих кодових баз з часом.

Продуктивність мови програмування Go є визначною особливістю. Це скомпільована мова, яка створює власний машинний код, що дозволяє створювати високоефективні та швидкодіючі додатки. Вбудована в Go модель паралелізму, Goroutines, дозволяє легко та ефективно програмувати паралельно, що робить її добре придатною для обробки високого трафіку та паралельних запитів у веб-додатках.

Що стосується фреймворків для веб-розробки, мова програмування Go пропонує кілька популярних варіантів. Одним з найпоширеніших фреймворків є стандартний пакет бібліотек Go ``net/http''. Він надає надійний набір інструментів для створення веб-серверів, управління маршрутизацією, обслуговування статичних файлів та управління HTTP-запитами і відповідями. Простота і потужність стандартної бібліотеки дозволяють розробникам створювати веб-додатки без необхідності важких зовнішніх залежностей.

Ще одним помітним фреймворком в екосистемі мови програмування Go є Gin. Gin - це легкий і високопродуктивний веб-фреймворк, який надає додаткові функції та абстракції на додаток до стандартної бібліотеки. Він пропонує швидкий маршрутизатор, підтримку проміжного програмного забезпечення та різноманітні утиліти для спрощення типових завдань веб-розробки.

Сильний акцент мови програмування Go на простоті та ефективності поширюється і на його розгортання. Додатки на Go компілюються в автономні двійкові файли, що спрощує розгортання та усуває необхідність у додаткових залежностях під час виконання. Ця характеристика робить Go добре придатною для створення мікросервісів або розгортання додатків на хмарних платформах.

Хоча мова програмування Go пропонує численні переваги для веб-розробки, важливо враховувати її відносну незрілість у порівнянні з більш усталеними мовами. Екосистема Go все ще розвивається, і доступність сторонніх бібліотек та інструментів може бути більш обмеженою порівняно з такими мовами, як Python або JavaScript. Проте, основна мова та стандартна бібліотека є стабільними та добре підтримуються спільнотою Go.

Отже, Go - це потужна мова програмування для веб-розробки, яка пропонує простоту, ефективність та високу продуктивність. Її чистий синтаксис, вбудована модель паралелізму та всеосяжна стандартна бібліотека роблять її чудовим вибором для створення масштабованих та високопродуктивних веб-додатків. Завдяки зростаючій екосистемі та підтримці спільноти, Go продовжує набирати популярність як найкраща мова для проектів веб-розробки.

Переваги Go (Golang):
\begin{itemize}
    \item Продуктивність: компільована природа Go та ефективна модель паралелізму сприяють створенню високопродуктивних веб-додатків.
    \item Спрощеність: чистий синтаксис та мінімалістичний дизайн Go сприяють читабельності коду та легкості його супроводу.
    \item Конкурентність: підпрограми та вбудовані примітиви паралельності спрощують паралельне програмування, дозволяючи ефективно обробляти декілька запитів.
    \item Стандартна бібліотека: стандартна бібліотека Go надає основні інструменти для веб-розробки, що полегшує початок роботи без важких зовнішніх залежностей.
    \item Розгортання: програми на Go компілюються у автономні двійкові файли, що спрощує розгортання і усуває необхідність у додаткових залежностях під час виконання.
\end{itemize}

Недоліки Go (Golang):
\begin{itemize}
    \item Зрілість екосистеми: екосистема Go є відносно молодою, і доступність сторонніх бібліотек та інструментів може бути більш обмеженою у порівнянні з більш усталеними мовами.
    \item Крива навчання: Go має власний набір парадигм і концепцій програмування, які можуть потребувати певного вивчення і пристосування для розробників, які не знайомі з мовою.
    \item Невелика кількість бібліотек: хоча екосистема Go розвивається, кількість доступних бібліотек і фреймворків може бути меншою у порівнянні з більш популярними мовами, такими як Python або JavaScript.
    \item Відсутність узагальнень: наразі Go не підтримує узагальнення, що іноді може призвести до дублювання коду або більш багатослівних рішень.
    \item розмір спільноти: хоча спільнота Go є активною і доброзичливою, вона може бути меншою порівняно зі спільнотами більш поширених мов.
\end{itemize}

\subsection{Остаточний вибів мови програмування}
\label{subsec:final-choice-subsection}

Python є ідеальним вибором для розробки даного застосунку, завдяки своїй універсальності, великим бібліотекам та простоті використання. Завдяки своїй багатій екосистемі пакетів, Python надає численні бібліотеки та фреймворки, які можуть значно полегшити пошук маршрутів та веб-розробку. Наявність таких фреймворків як Django або Flask для веб-розробки є надійними та ефективними інструменти. Важливим є те, що мова програмування Python є популярним вибором для роботи з даними та графами, що значно полегшує роботу над частиною проекту з пошуком маршруту. Крім того, простий і зрозумілий синтаксис Python дозволяє пришвидшити розробку та полегшити підтримку коду. Велика та активна спільнота цієї мови програмування означає, що існує безліч ресурсів, навчальних посібників та підтримки для подолання будь-яких проблем, з якими можна зіткнутися в процесі розробки. Крім того, сумісність Python з іншими технологіями та платформами робить цю мову програмування підходящим вибором для створення масштабованих та крос-платформних додатків. Загалом, поєднання потужності, гнучкості та підтримки спільноти робить Python найкращим вибором для розробки додатку для пошуку маршрутів.