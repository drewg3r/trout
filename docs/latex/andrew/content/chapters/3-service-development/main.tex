\chapter{Розробка застосунку}
\label{chap:development}

Розробка веб-сервісу, який надає можливості пошуку маршрутів, є серйозним завданням, що вимагає ретельного планування, ефективної реалізації та безперешкодної інтеграції різних технологій. У сучасному швидкоплинному світі, де подорожі та транспорт відіграють вирішальну роль у нашому повсякденному житті, потреба в ефективному та надійному рішенні для пошуку маршрутів є першочерговою. Незалежно від того, чи це стосується пасажирів, які шукають найкоротший шлях до місця призначення, чи туристів, які досліджують незнайомі міста, добре розроблений веб-сервіс може спростити процес пошуку оптимальних маршрутів і покращити загальний користувацький досвід.

Цей розділ заглиблюється у сферу розробки веб-сервісу, досліджуючи тонкощі, пов'язані зі створенням надійних, масштабованих і зручних для користувачів онлайн-сервісів. Ми заглибимося у фундаментальні концепції, методології та технології, які лежать в основі розробки сучасних веб-сервісів. Ця глава має на меті забезпечити комплексне розуміння процесу розробки веб-сервісів - від дизайну та архітектури до стратегій впровадження та розгортання.

У ході вивчення цього розділу ми розглянемо різні аспекти розробки веб-сервісів, включаючи вибір відповідних мов програмування, фреймворків та інструментів, принципи проектування для створення інтуїтивно зрозумілих користувацьких інтерфейсів, реалізацію безпечних механізмів автентифікації та авторизації, а також стратегії розгортання для забезпечення оптимальної продуктивності та масштабованості. Крім того, ми торкнемося таких ключових аспектів, як управління даними, інтеграція API та методології тестування, які є важливими для створення надійного та ефективного веб-сервісу.

Отримавши уявлення про тонкощі розробки веб-сервісів, ми зможемо ефективно планувати, розробляти та впроваджувати надійні та орієнтовані на користувача веб-сервіси, які задовольнятимуть потреби нашої цільової аудиторії. Знання та розуміння, отримані з цього розділу, забезпечать нас необхідними інструментами та методами, щоб орієнтуватися в постійно мінливому ландшафті розробки веб-сервісів і надавати інноваційні та ефективні послуги в Інтернеті.

\section{Загальний опис роботи системи, її можливості}
\label{sec:functionality}

Система планування маршрутів, що розроблюється, є складною та інтуїтивно зрозумілою, полегшує планування маршрутів та навігацію для користувачів. Вона пропонує широкий спектр функцій та інструментів, забезпечуючи ефективний та безперебійний пошук оптимальних маршрутів різними видами транспорту.

В основі системи лежить потужна функція пошуку маршрутів. Користувачі можуть ввести початкову точку і пункт призначення, а система планування маршрутів швидко розрахує і представить найбільш підходящий маршрут на основі різних факторів, таких як час у дорозі, час у транспорті та час прибуття.

Підтримка різних видів транспорту та різноманітних маршрутів є ключовою перевагою системи. Адміністратор транспортної компанії може додавати будь-які транспортні мартрути. Ця різноманітність дозволяє пропонувати індивідуальні маршрути, які відповідають індивідуальним уподобанням та конкретним вимогам до подорожей.

Дані, надані адміністратором чи представником транспортної компанії, використовуютьчся для того щоб ефективно та зручно для користувача надавати актуальну інформацію про умови дорожнього руху, розклад громадського транспорту та додаткову інформацію про маршрути.

Персоналізація - ще один важливий аспект системи. Адміністратор чи представник транспортної компанії може додати будь-який вид маршруту, такий як прямий маршрут, маршрут з проміжними зупинками, кільцевий маршрут, чи маршрут який складається з багатьох відрізків. Також маршрутам можна задавати будь-який разклад, як регулярні та часті відправлення, так і поїздки які відбуваються раз на місяць.


Завдяки всеосяжній базі даних про визначні місця, система планування маршрутів дозволяє користувачам досліджувати пам'ятки, визначні місця, ресторани та інші об'єкти, розташовані вздовж їхніх маршрутів. Ця багата інтеграція покращує процес прийняття рішень і полегшує навігацію, надаючи контекстну інформацію та відповідні рекомендації.

Візуальне представлення є важливим компонентом інтерфейсу системи. Користувачі можуть побачити на карті місцезнаходження будь-якої обраної зупинки, серед іншої інформації про зупинку. Користувачі можуть легко збільшувати та зменшувати масштаб, панорамувати карту та отримувати доступ до додаткової інформації за потреби.

Визнаючи важливість мобільного доступу, система планування маршрутів має адаптивний дизайн для сумісності з різними пристроями, включаючи настільні комп'ютери, смартфони та планшети. Така адаптивність гарантує, що користувачі можуть отримати доступ до її потужних функцій і використовувати їх на ходу, зміцнюючи її статус як зручного і незамінного навігаційного інструменту.

Загалом, система планування маршрутів втілює прагнення спростити процес планування маршрутів, надаючи точні та ефективні навігаційні рішення. Поєднуючи передові алгоритми, інтеграцію даних у режимі реального часу та зручні інтерфейси, вона створює передумови для безперебійного та ефективного планування маршрутів, що задовольняє різноманітні потреби користувачів.


%Subsections
\subsection{Основний функціонал системи}
\label{subsec:features-subsection}

Сервіс пошуку маршрутів, який розроблюється, пропонує широкий спектр функцій, покликаних спростити та покращити досвід подорожей для наших користувачів. Незалежно від того, чи пересуваються вони жвавими міськими вулицями, чи планують щоденні поїздки на роботу, чи вирушають у нову подорож, цей сервіс надає інструменти та функціональність, які допоможуть їм знайти найкращі маршрути відповідно до їхніх уподобань. Від оновлень трафіку в реальному часі до персоналізованого планування маршрутів - нижче нанедені деякі з ключових функцій, які пропонує даний сервіс пошуку маршрутів транспорту.

Перелік основних функцій доступних в системі:
\begin{itemize}
    \item Пошук маршруту: Можливість знайти найкоротший або оптимальний маршрут між двома або більше місцями на основі різних критеріїв, таких як відстань, час у дорозі, вид транспорту або вподобання користувача.
    \item Різні види транспорту: Підтримка різних видів транспорту, включаючи громадський транспорт (автобус, потяг, метро), автомобіль, пішохідний, велосипедний та інші, що дозволяє користувачам обирати найбільш підходящий спосіб пересування для своєї подорожі.
    \item Маршрути з кількома зупинками та пунктами призначення: Підтримка пошуку та відображення маршрутів, які складаються з кількох видів транспорту.
    \item Покрокові вказівки: Покрокові вказівки та інструкції, на протязі всього прокладеного маршруту, включаючи інформацію про час відправлення, проміжні зупинки, станції для пересадки, карта з розміщенням зупинки.
    \item Сумісність: Сумісність з різними браузерами та пристроями, включаючи настільні комп'ютери, смартфони, планшети та що дозволяє користувачам отримувати доступ до сервісу з їхніх улюблених пристроїв.
\end{itemize}

Ці та багато інших функцій роблять сервіс для пошуку маршрутів, що розроблюється, цінним інструментом для людей, які шукають ефективні та оптимізовані маршрути для щоденних поїздок на роботу, подорожей та інших цілей. Завдяки зручному інтерфейсу та розширеній функціональності можна забезпечити безперебійну та приємну роботу для всіх користувачів.

\subsection{Отримання, обробка та відображення даних про транспорт}
\label{subsec:route-management-subsection}

В основі сервісу пошуку маршрутів лежить ефективна обробка та використання транспортних даних. У цьому розділі буде заглиблено у складний процес отримання, обробки та відображення транспортних даних, який є основою функціональності сервісу для пошуку маршрутів. 

Для надання точної та актуальної транспортної інформації в сервісі використовується інтерактивний користувацький інтерфейс, за допомогою якого адміністратор чи представник транспортної компанії може вводити дані про транспортні маршрути, щоб колристувачі могли ними користуватись. Також інтерфейс дозволяє користувачам шукати маршрути, що підходять їх потребам, вказавши бажаний час відправлення, початкову зупинку та пункт призначення.

Після отримання даних з інтерфейсу система для пошуку маршрутів використовує складні алгоритми і методи обробки даних для перетворення введених користувачем даних у придатний для використання формат. Це включає перевірку введених даних, зіставлення їх з наявними транспортними даними та визначення найкращих можливих маршрутів на основі вподобань користувача.

Оброблені дані потім організовуються в комплексну і взаємопов'язану транспортну мережу, де встановлюються зв'язки між різними маршрутами, зупинками і пунктами призначення. Це дозволяє сервісу генерувати точні та ефективні маршрути на основі даних та вподобань користувачів. Транспортні дані доповнюються додатковою контекстною інформацією, такою як дані про об'єкти, орієнтири та відповідні геопросторові дані, що збагачує загальний користувацький досвід.

Відображення оброблених транспортних даних у зручний та візуально привабливий спосіб має вирішальне значення для забезпечення безперешкодної роботи користувачів. В даному сервісі використовуються сучасні методи візуалізації даних та інтерактивні карти для представлення зупинок. Інтерфейс розроблений таким чином, щоб бути інтуїтивно зрозумілим і зручним для навігації, що дозволяє користувачам легко досліджувати та взаємодіяти з транспортними даними.

Зображення сторінки для додавання даних

Зораження сторінки для відображення доданих маршрутів


Щоб зробити можливим додавання транспортних маршрутів до системи пошуку маршрутів, потрібно реалізували функціонал, який дозволяє авторизованим адміністраторам чи відповідальним особам додавати нові маршрути. Використовуючи архітектуру Django Model-View-Controller (MVC), потрібно створити спеціальну функцію для відображення транспортних маршрутів та форму для створення транспортного маршруту. Форма збирає необхідну інформацію для створення маршруту. Після відправки даних, вони перевіряються спеціальною функцією і створюється новий об'єкт маршруту в базі даних, використовуючи ORM Django. Це гарантує, що доданий маршрут зберігається постійно і до нього можна отримати доступ для подальшої обробки та відображення.

Обробка транспортних маршрутів включає в себе маніпуляції та аналіз даних маршруту для отримання змістовної інформації. Для цьог використовуються потужні можливості запитів Django для отримання відповідних маршрутів з бази даних на основі певних критеріїв. Наприклад, можна отримати маршрути, які задовольняють певним параметрам, таким як відстань або час. Маючи отримані дані про маршрут, можна виконувати розрахунки і застосовувати алгоритми для визначення таких факторів, як найкоротший маршрут, оптимальний розклад або альтернативні маршрути. Ці оброблені результати можна зберігати або використовувати для подальших операцій, таких як рекомендації щодо маршруту або оптимізація розкладу.

Щоб надати користувачам інтуїтивно зрозуміле та інформативне відображення транспортних маршрутів, використовується система шаблонів Django та HTML/CSS для створення динамічних і візуально привабливих веб-сторінок. За допомогою цього розробляються шаблони, які ефективно представляють деталі маршруту, включаючи пункт відправлення, пункт призначення, проміжні зупинки, пересадки, час у дорозі та види транспорту.

Реалізувавши ці функції в рамках проекту на Python та Django, можна легко обробляти додавання, обробку та відображення транспортних маршрутів. Це забезпечує ефективне управління даними, точні розрахунки маршрутів і візуально привабливий користувальницький інтерфейс, що дозволяє користувачам приймати обґрунтовані рішення щодо подорожей на основі доступних варіантів транспорту.

\subsection{Пошук маршруту}
\label{subsec:route-search-subsection}

Сторінка пошуку маршрутів є основною функцією сервісу для пошуку маршрутів, надаючи користувачам зручний та інтуїтивно зрозумілий інтерфейс для планування своїх подорожей. У цьому розділі розглядаються ключові функції та можливості сторінки для пошуку маршрутів.

По суті, сторінка для пошуку маршрутів пропонує користувачам можливість легко планувати свої маршрути з одного місця в інше. Користувачі можуть ввести початкову точку і пункт призначення, а сервіс для пошуку маршрутів, використовуючи складні алгоритми і методи обробки даних, генерує оптимальні маршрути відповідно до запитів користувачів. Цей спрощений процес позбавляє користувачів необхідності орієнтуватися в складних транспортних мережах або вручну шукати маршрути найкоротші маршрути серед купи транспорту.

Однією з основних причин, чому користувачі покладаються на сторінку пошуку маршрутів, є її здатність економити час і зусилля. Завдяки комплексному огляду доступних маршрутів, включаючи різні види транспорту, альтернативні шляхи та орієнтовний час у дорозі, користувачі можуть приймати обґрунтовані рішення без необхідності проведення тривалих досліджень або спроб і помилок. Інтуїтивно зрозумілий дизайн сторінки та зручний інтерфейс роблять її доступною для користувачів з будь-якою технічною підготовкою, що ще більше підвищує важливість сторінки для пошуку маршрутів, її привабливість та зручність використання.

Окрім економії часу та зусиль, сторінка для пошуку маршрутів пропонує користувачам можливість відкрити для себе нові та ефективні способи навігації в транспортній мережі. Вона представляє користувачам низку транспортних варіантів. Це дозволяє користувачам обирати вид транспорту, який найкраще відповідає їхнім потребам, беручи до уваги такі фактори, як зручність, вартість, вплив на навколишнє середовище та особисті уподобання. Динамічний та адаптивний характер сторінки гарантує, що користувачі можуть знайти маршрути, які відповідають їхнім конкретним вимогам.

Крім того, сторінка пошуку маршрутів слугує надійним помічником для користувачів, які покладаються на мережі громадського транспорту. Вона надає детальну інформацію про розклад руху, зупинки та пересадки, що дозволяє користувачам точно і впевнено планувати свої подорожі.

Таким чином, сторінка для пошуку маршрутів пропонує користувачам комплексний та ефективний інструмент для планування своїх поїздок. Спрощуючи процес пошуку маршруту, вона дозволяє користувачам приймати обґрунтовані рішення, економити час і знаходити оптимальні маршрути, які відповідають їхнім уподобанням. Зручний інтерфейс, гнучкість та інтеграція з даними в режимі реального часу роблять сервіс для пошуку маршрутів незамінним ресурсом для людей, які шукають швидку, зручну та надійну навігацію у своїх повсякденних подорожах.

Зображення сторінки для пошуку маршрутів


Для того щоб розробити вище описану сторінку, для розробки фронтенду потрібно зробити інтуїтивно зрозумілий користувацький інтерфейс, який дозволяє користувачам вводити бажані станції відправлення та призначення. Для цього використовуються сучасні веб-технології, такі як HTML, CSS та JavaScript, щоб створити візуально привабливий та адаптивний інтерфейс.

Внутрішня частина сторінки пошуку маршрутів відіграє життєво важливу роль в обробці даних, введених користувачем, і отриманні відповідних даних. Використовуючи фреймворк Django, буде розроблено необхідні функції для представлення та обробники для обробки вхідних запитів та ініціювання процесу пошуку маршруту. Ці функції для представлення витягують дані, введені користувачем, і викликають логіку серверної частини, щоб знайти найбільш підходящий маршрут на основі попередньо визначених алгоритмів і моделей даних.

Бекенд використовує надійні можливості Django для отримання даних з основної бази даних. Завдяки чітко визначеним моделям, що представляють станції, маршрути, сполучення та точки маршруту, сервіс може ефективно отримувати необхідну інформацію для розрахунку маршруту. Використовуючи ORM (об'єктно-реляційне відображення) Django, легко долається розрив між операціями з базою даних та написаним кодом на Python.

Для визначення оптимального маршруту застосовується алгоритм, який ретельно враховує такі фактори, як відстань, час та бажані види транспорту. Бекенд обробляє дані, виконує необхідні обчислення і повертає результати на фронтенд для відображення.

Завдяки такій злагодженій взаємодії фронтенду та бекенду користувачі можуть легко шукати маршрути, переглядати відповідні деталі та приймати обґрунтовані рішення щодо подорожі. Сторінка для пошуку маршрутів є найголовнішою сторінкою застосунку, поєднуючи елегантний користувальницький інтерфейс з надійною функціональністю бекенду для надання точних та ефективних пропозицій щодо маршрутів.

\subsection{Відображення знайденого маршруту}
\label{subsec:route-displaying-subsection}

Сторінка для відображення знайденого маршруту є важливим компонентом сервысу для пошуку маршрутів, надаючи користувачам комплексне уявлення про підібрані для них маршрути. У цьому розділі розглядаються функції та можливості сторінки для відображення знайденого маршруту.

Основна функція сторінки для відображення знайденого маршруту - надати користувачам візуально привабливе та інформативне представлення підібраних для них маршрутів. Після того, як користувачі здійснили пошук маршруту, на цій сторінці відображається знайдений маршрут разом з різними деталями та інтерактивними елементами. Користувачі можуть переглядати покрокові вказівки, карти з зупинками, пересадки, приблизний час у дорозі та іншу інформацію, яка допомагає їм ефективно орієнтуватися під час подорожі.

Однією з ключових причин, чому користувачі покладаються на сторінку відображення маршруту, є її здатність забезпечити чітке розуміння всього маршруту від початку до кінця. Вона представляє користувачам огляд прокладеного маршруту, виділяючи послідовність маршрутних точок, об'єктів і будь-яких необхідних змін режиму руху. Таке комплексне відображення дає змогу користувачам підготуватися до подорожі, забезпечуючи цілісне розуміння майбутнього маршруту та передбачаючи будь-які потенційні виклики або орієнтири на шляху.

Крім того, сторінка відображення маршруту пропонує користувачам низку інтерактивних функцій для покращення вивчення маршруту. Користувачі можуть збільшувати і зменшувати масштаб мапи і досліджувати конкретні об'єкти або орієнтири вздовж маршруту. Ці інтерактивні елементи надають користувачам динамічний і цікавий досвід, дозволяючи їм ознайомитися з навколишнім середовищем, визначити орієнтири і приймати обґрунтовані рішення під час подорожі.

Крім того, на сторінці відображення маршруту пріоритетом є доступність та зручність для користувачів. Вона розроблена таким чином, щоб бути адаптивною та зрозумілою, забезпечуючи безперебійний перегляд на різних пристроях та екранах різного розміру. Це дозволяє користувачам отримувати доступ до обраних маршрутів в такому вигляді, в якому їм зручно, і взаємодіяти з ними, незалежно від того, чи використовують вони стаціонарний комп'ютер, планшет, мобільний пристрій чи будь-який інший пристрій, де є браузер.

Таким чином, сторінка для відображення знайденого маршруту відіграє вирішальну роль у наданні користувачам комплексного та інтерактивного відображення обраних ними маршрутів. Пропонуючи детальні вказівки, інтерактивні карти, вона дозволяє користувачам впевнено та ефективно орієнтуватися у транспортних мережах під час подорожі, чи на етапі її планування. Зручний інтерфейс, цікаві інтерактивні елементи та доступність роблять сторінку для відображення знайденого маршруту цінним ресурсом для тих, хто шукає зручне та інформативне відображення маршрутів.


Зображення сторінки для пошуку маршрутів


Щоб реалізувати функцію відображення маршруту у нашому веб-додатку, ми використовуємо потужні можливості та гнучкість фреймворку Django. В основі реалізації лежить використання представлень і шаблонів Django. Ми створюємо спеціальне представлення, яке отримує обрану інформацію про маршрут з бекенду і відображає її за допомогою відповідного шаблону. Представлення отримує необхідні дані, такі як покрокові інструкції, точки маршруту, орієнтовний час у дорозі та інші відповідні деталі, з нашої внутрішньої бази даних або зовнішніх API.

Система шаблонів Django дозволяє нам структурувати і форматувати інформацію про маршрут у візуально привабливий спосіб. За допомогою використання HTML, CSS та мови шаблонів Django можна створити зручний та функціональний інтерфейс, який представляє деталі маршруту в чіткій, зручній, зрозумілій та організованій формі. Шаблон може включє інтерактивні елементи, такі як карти, функцію масштабування та перемикання між різними видами карт, щоб покращити користувацький досвід.

Крім того, система автентифікації та авторизації користувачів Django дозволяє представникам транспортної компанії керувати доступом до управляння транспортними даними і надавати доступ до цієї функції лише відповідальним особам. Це дозволяє відділити управління транспортом від функцій доступних звичайним користувачам, таким як перегляд маршрутів, пошук маршрутів і інших.

Крім того, підтримка Django адаптивного веб-дизайну та мобільної оптимізації гарантує, що сторінка з маршрутом буде доступною і функціональною на різних пристроях і з різними розмірами екрану. Використовуючи адаптивні CSS фреймворки, такі як Bootstrap, ми можемо досягти адаптивного і візуально привабливого інтерфейсу користувача, який адаптується до пристрою користувача, будь то настільний комп'ютер, планшет або мобільний телефон.

\section{Загальний опис роботи системи, її можливості}
\label{sec:functionality}

Система планування маршрутів, що розроблюється, є складною та інтуїтивно зрозумілою, полегшує планування маршрутів та навігацію для користувачів. Вона пропонує широкий спектр функцій та інструментів, забезпечуючи ефективний та безперебійний пошук оптимальних маршрутів різними видами транспорту.

В основі системи лежить потужна функція пошуку маршрутів. Користувачі можуть ввести початкову точку і пункт призначення, а система планування маршрутів швидко розрахує і представить найбільш підходящий маршрут на основі різних факторів, таких як час у дорозі, час у транспорті та час прибуття.

Підтримка різних видів транспорту та різноманітних маршрутів є ключовою перевагою системи. Адміністратор транспортної компанії може додавати будь-які транспортні мартрути. Ця різноманітність дозволяє пропонувати індивідуальні маршрути, які відповідають індивідуальним уподобанням та конкретним вимогам до подорожей.

Дані, надані адміністратором чи представником транспортної компанії, використовуютьчся для того щоб ефективно та зручно для користувача надавати актуальну інформацію про умови дорожнього руху, розклад громадського транспорту та додаткову інформацію про маршрути.

Персоналізація - ще один важливий аспект системи. Адміністратор чи представник транспортної компанії може додати будь-який вид маршруту, такий як прямий маршрут, маршрут з проміжними зупинками, кільцевий маршрут, чи маршрут який складається з багатьох відрізків. Також маршрутам можна задавати будь-який разклад, як регулярні та часті відправлення, так і поїздки які відбуваються раз на місяць.


Завдяки всеосяжній базі даних про визначні місця, система планування маршрутів дозволяє користувачам досліджувати пам'ятки, визначні місця, ресторани та інші об'єкти, розташовані вздовж їхніх маршрутів. Ця багата інтеграція покращує процес прийняття рішень і полегшує навігацію, надаючи контекстну інформацію та відповідні рекомендації.

Візуальне представлення є важливим компонентом інтерфейсу системи. Користувачі можуть побачити на карті місцезнаходження будь-якої обраної зупинки, серед іншої інформації про зупинку. Користувачі можуть легко збільшувати та зменшувати масштаб, панорамувати карту та отримувати доступ до додаткової інформації за потреби.

Визнаючи важливість мобільного доступу, система планування маршрутів має адаптивний дизайн для сумісності з різними пристроями, включаючи настільні комп'ютери, смартфони та планшети. Така адаптивність гарантує, що користувачі можуть отримати доступ до її потужних функцій і використовувати їх на ходу, зміцнюючи її статус як зручного і незамінного навігаційного інструменту.

Загалом, система планування маршрутів втілює прагнення спростити процес планування маршрутів, надаючи точні та ефективні навігаційні рішення. Поєднуючи передові алгоритми, інтеграцію даних у режимі реального часу та зручні інтерфейси, вона створює передумови для безперебійного та ефективного планування маршрутів, що задовольняє різноманітні потреби користувачів.


%Subsections
\subsection{Основний функціонал системи}
\label{subsec:features-subsection}

Сервіс пошуку маршрутів, який розроблюється, пропонує широкий спектр функцій, покликаних спростити та покращити досвід подорожей для наших користувачів. Незалежно від того, чи пересуваються вони жвавими міськими вулицями, чи планують щоденні поїздки на роботу, чи вирушають у нову подорож, цей сервіс надає інструменти та функціональність, які допоможуть їм знайти найкращі маршрути відповідно до їхніх уподобань. Від оновлень трафіку в реальному часі до персоналізованого планування маршрутів - нижче нанедені деякі з ключових функцій, які пропонує даний сервіс пошуку маршрутів транспорту.

Перелік основних функцій доступних в системі:
\begin{itemize}
    \item Пошук маршруту: Можливість знайти найкоротший або оптимальний маршрут між двома або більше місцями на основі різних критеріїв, таких як відстань, час у дорозі, вид транспорту або вподобання користувача.
    \item Різні види транспорту: Підтримка різних видів транспорту, включаючи громадський транспорт (автобус, потяг, метро), автомобіль, пішохідний, велосипедний та інші, що дозволяє користувачам обирати найбільш підходящий спосіб пересування для своєї подорожі.
    \item Маршрути з кількома зупинками та пунктами призначення: Підтримка пошуку та відображення маршрутів, які складаються з кількох видів транспорту.
    \item Покрокові вказівки: Покрокові вказівки та інструкції, на протязі всього прокладеного маршруту, включаючи інформацію про час відправлення, проміжні зупинки, станції для пересадки, карта з розміщенням зупинки.
    \item Сумісність: Сумісність з різними браузерами та пристроями, включаючи настільні комп'ютери, смартфони, планшети та що дозволяє користувачам отримувати доступ до сервісу з їхніх улюблених пристроїв.
\end{itemize}

Ці та багато інших функцій роблять сервіс для пошуку маршрутів, що розроблюється, цінним інструментом для людей, які шукають ефективні та оптимізовані маршрути для щоденних поїздок на роботу, подорожей та інших цілей. Завдяки зручному інтерфейсу та розширеній функціональності можна забезпечити безперебійну та приємну роботу для всіх користувачів.

\subsection{Отримання, обробка та відображення даних про транспорт}
\label{subsec:route-management-subsection}

В основі сервісу пошуку маршрутів лежить ефективна обробка та використання транспортних даних. У цьому розділі буде заглиблено у складний процес отримання, обробки та відображення транспортних даних, який є основою функціональності сервісу для пошуку маршрутів. 

Для надання точної та актуальної транспортної інформації в сервісі використовується інтерактивний користувацький інтерфейс, за допомогою якого адміністратор чи представник транспортної компанії може вводити дані про транспортні маршрути, щоб колристувачі могли ними користуватись. Також інтерфейс дозволяє користувачам шукати маршрути, що підходять їх потребам, вказавши бажаний час відправлення, початкову зупинку та пункт призначення.

Після отримання даних з інтерфейсу система для пошуку маршрутів використовує складні алгоритми і методи обробки даних для перетворення введених користувачем даних у придатний для використання формат. Це включає перевірку введених даних, зіставлення їх з наявними транспортними даними та визначення найкращих можливих маршрутів на основі вподобань користувача.

Оброблені дані потім організовуються в комплексну і взаємопов'язану транспортну мережу, де встановлюються зв'язки між різними маршрутами, зупинками і пунктами призначення. Це дозволяє сервісу генерувати точні та ефективні маршрути на основі даних та вподобань користувачів. Транспортні дані доповнюються додатковою контекстною інформацією, такою як дані про об'єкти, орієнтири та відповідні геопросторові дані, що збагачує загальний користувацький досвід.

Відображення оброблених транспортних даних у зручний та візуально привабливий спосіб має вирішальне значення для забезпечення безперешкодної роботи користувачів. В даному сервісі використовуються сучасні методи візуалізації даних та інтерактивні карти для представлення зупинок. Інтерфейс розроблений таким чином, щоб бути інтуїтивно зрозумілим і зручним для навігації, що дозволяє користувачам легко досліджувати та взаємодіяти з транспортними даними.

Зображення сторінки для додавання даних

Зораження сторінки для відображення доданих маршрутів


Щоб зробити можливим додавання транспортних маршрутів до системи пошуку маршрутів, потрібно реалізували функціонал, який дозволяє авторизованим адміністраторам чи відповідальним особам додавати нові маршрути. Використовуючи архітектуру Django Model-View-Controller (MVC), потрібно створити спеціальну функцію для відображення транспортних маршрутів та форму для створення транспортного маршруту. Форма збирає необхідну інформацію для створення маршруту. Після відправки даних, вони перевіряються спеціальною функцією і створюється новий об'єкт маршруту в базі даних, використовуючи ORM Django. Це гарантує, що доданий маршрут зберігається постійно і до нього можна отримати доступ для подальшої обробки та відображення.

Обробка транспортних маршрутів включає в себе маніпуляції та аналіз даних маршруту для отримання змістовної інформації. Для цьог використовуються потужні можливості запитів Django для отримання відповідних маршрутів з бази даних на основі певних критеріїв. Наприклад, можна отримати маршрути, які задовольняють певним параметрам, таким як відстань або час. Маючи отримані дані про маршрут, можна виконувати розрахунки і застосовувати алгоритми для визначення таких факторів, як найкоротший маршрут, оптимальний розклад або альтернативні маршрути. Ці оброблені результати можна зберігати або використовувати для подальших операцій, таких як рекомендації щодо маршруту або оптимізація розкладу.

Щоб надати користувачам інтуїтивно зрозуміле та інформативне відображення транспортних маршрутів, використовується система шаблонів Django та HTML/CSS для створення динамічних і візуально привабливих веб-сторінок. За допомогою цього розробляються шаблони, які ефективно представляють деталі маршруту, включаючи пункт відправлення, пункт призначення, проміжні зупинки, пересадки, час у дорозі та види транспорту.

Реалізувавши ці функції в рамках проекту на Python та Django, можна легко обробляти додавання, обробку та відображення транспортних маршрутів. Це забезпечує ефективне управління даними, точні розрахунки маршрутів і візуально привабливий користувальницький інтерфейс, що дозволяє користувачам приймати обґрунтовані рішення щодо подорожей на основі доступних варіантів транспорту.

\subsection{Пошук маршруту}
\label{subsec:route-search-subsection}

Сторінка пошуку маршрутів є основною функцією сервісу для пошуку маршрутів, надаючи користувачам зручний та інтуїтивно зрозумілий інтерфейс для планування своїх подорожей. У цьому розділі розглядаються ключові функції та можливості сторінки для пошуку маршрутів.

По суті, сторінка для пошуку маршрутів пропонує користувачам можливість легко планувати свої маршрути з одного місця в інше. Користувачі можуть ввести початкову точку і пункт призначення, а сервіс для пошуку маршрутів, використовуючи складні алгоритми і методи обробки даних, генерує оптимальні маршрути відповідно до запитів користувачів. Цей спрощений процес позбавляє користувачів необхідності орієнтуватися в складних транспортних мережах або вручну шукати маршрути найкоротші маршрути серед купи транспорту.

Однією з основних причин, чому користувачі покладаються на сторінку пошуку маршрутів, є її здатність економити час і зусилля. Завдяки комплексному огляду доступних маршрутів, включаючи різні види транспорту, альтернативні шляхи та орієнтовний час у дорозі, користувачі можуть приймати обґрунтовані рішення без необхідності проведення тривалих досліджень або спроб і помилок. Інтуїтивно зрозумілий дизайн сторінки та зручний інтерфейс роблять її доступною для користувачів з будь-якою технічною підготовкою, що ще більше підвищує важливість сторінки для пошуку маршрутів, її привабливість та зручність використання.

Окрім економії часу та зусиль, сторінка для пошуку маршрутів пропонує користувачам можливість відкрити для себе нові та ефективні способи навігації в транспортній мережі. Вона представляє користувачам низку транспортних варіантів. Це дозволяє користувачам обирати вид транспорту, який найкраще відповідає їхнім потребам, беручи до уваги такі фактори, як зручність, вартість, вплив на навколишнє середовище та особисті уподобання. Динамічний та адаптивний характер сторінки гарантує, що користувачі можуть знайти маршрути, які відповідають їхнім конкретним вимогам.

Крім того, сторінка пошуку маршрутів слугує надійним помічником для користувачів, які покладаються на мережі громадського транспорту. Вона надає детальну інформацію про розклад руху, зупинки та пересадки, що дозволяє користувачам точно і впевнено планувати свої подорожі.

Таким чином, сторінка для пошуку маршрутів пропонує користувачам комплексний та ефективний інструмент для планування своїх поїздок. Спрощуючи процес пошуку маршруту, вона дозволяє користувачам приймати обґрунтовані рішення, економити час і знаходити оптимальні маршрути, які відповідають їхнім уподобанням. Зручний інтерфейс, гнучкість та інтеграція з даними в режимі реального часу роблять сервіс для пошуку маршрутів незамінним ресурсом для людей, які шукають швидку, зручну та надійну навігацію у своїх повсякденних подорожах.

Зображення сторінки для пошуку маршрутів


Для того щоб розробити вище описану сторінку, для розробки фронтенду потрібно зробити інтуїтивно зрозумілий користувацький інтерфейс, який дозволяє користувачам вводити бажані станції відправлення та призначення. Для цього використовуються сучасні веб-технології, такі як HTML, CSS та JavaScript, щоб створити візуально привабливий та адаптивний інтерфейс.

Внутрішня частина сторінки пошуку маршрутів відіграє життєво важливу роль в обробці даних, введених користувачем, і отриманні відповідних даних. Використовуючи фреймворк Django, буде розроблено необхідні функції для представлення та обробники для обробки вхідних запитів та ініціювання процесу пошуку маршруту. Ці функції для представлення витягують дані, введені користувачем, і викликають логіку серверної частини, щоб знайти найбільш підходящий маршрут на основі попередньо визначених алгоритмів і моделей даних.

Бекенд використовує надійні можливості Django для отримання даних з основної бази даних. Завдяки чітко визначеним моделям, що представляють станції, маршрути, сполучення та точки маршруту, сервіс може ефективно отримувати необхідну інформацію для розрахунку маршруту. Використовуючи ORM (об'єктно-реляційне відображення) Django, легко долається розрив між операціями з базою даних та написаним кодом на Python.

Для визначення оптимального маршруту застосовується алгоритм, який ретельно враховує такі фактори, як відстань, час та бажані види транспорту. Бекенд обробляє дані, виконує необхідні обчислення і повертає результати на фронтенд для відображення.

Завдяки такій злагодженій взаємодії фронтенду та бекенду користувачі можуть легко шукати маршрути, переглядати відповідні деталі та приймати обґрунтовані рішення щодо подорожі. Сторінка для пошуку маршрутів є найголовнішою сторінкою застосунку, поєднуючи елегантний користувальницький інтерфейс з надійною функціональністю бекенду для надання точних та ефективних пропозицій щодо маршрутів.

\subsection{Відображення знайденого маршруту}
\label{subsec:route-displaying-subsection}

Сторінка для відображення знайденого маршруту є важливим компонентом сервысу для пошуку маршрутів, надаючи користувачам комплексне уявлення про підібрані для них маршрути. У цьому розділі розглядаються функції та можливості сторінки для відображення знайденого маршруту.

Основна функція сторінки для відображення знайденого маршруту - надати користувачам візуально привабливе та інформативне представлення підібраних для них маршрутів. Після того, як користувачі здійснили пошук маршруту, на цій сторінці відображається знайдений маршрут разом з різними деталями та інтерактивними елементами. Користувачі можуть переглядати покрокові вказівки, карти з зупинками, пересадки, приблизний час у дорозі та іншу інформацію, яка допомагає їм ефективно орієнтуватися під час подорожі.

Однією з ключових причин, чому користувачі покладаються на сторінку відображення маршруту, є її здатність забезпечити чітке розуміння всього маршруту від початку до кінця. Вона представляє користувачам огляд прокладеного маршруту, виділяючи послідовність маршрутних точок, об'єктів і будь-яких необхідних змін режиму руху. Таке комплексне відображення дає змогу користувачам підготуватися до подорожі, забезпечуючи цілісне розуміння майбутнього маршруту та передбачаючи будь-які потенційні виклики або орієнтири на шляху.

Крім того, сторінка відображення маршруту пропонує користувачам низку інтерактивних функцій для покращення вивчення маршруту. Користувачі можуть збільшувати і зменшувати масштаб мапи і досліджувати конкретні об'єкти або орієнтири вздовж маршруту. Ці інтерактивні елементи надають користувачам динамічний і цікавий досвід, дозволяючи їм ознайомитися з навколишнім середовищем, визначити орієнтири і приймати обґрунтовані рішення під час подорожі.

Крім того, на сторінці відображення маршруту пріоритетом є доступність та зручність для користувачів. Вона розроблена таким чином, щоб бути адаптивною та зрозумілою, забезпечуючи безперебійний перегляд на різних пристроях та екранах різного розміру. Це дозволяє користувачам отримувати доступ до обраних маршрутів в такому вигляді, в якому їм зручно, і взаємодіяти з ними, незалежно від того, чи використовують вони стаціонарний комп'ютер, планшет, мобільний пристрій чи будь-який інший пристрій, де є браузер.

Таким чином, сторінка для відображення знайденого маршруту відіграє вирішальну роль у наданні користувачам комплексного та інтерактивного відображення обраних ними маршрутів. Пропонуючи детальні вказівки, інтерактивні карти, вона дозволяє користувачам впевнено та ефективно орієнтуватися у транспортних мережах під час подорожі, чи на етапі її планування. Зручний інтерфейс, цікаві інтерактивні елементи та доступність роблять сторінку для відображення знайденого маршруту цінним ресурсом для тих, хто шукає зручне та інформативне відображення маршрутів.


Зображення сторінки для пошуку маршрутів


Щоб реалізувати функцію відображення маршруту у нашому веб-додатку, ми використовуємо потужні можливості та гнучкість фреймворку Django. В основі реалізації лежить використання представлень і шаблонів Django. Ми створюємо спеціальне представлення, яке отримує обрану інформацію про маршрут з бекенду і відображає її за допомогою відповідного шаблону. Представлення отримує необхідні дані, такі як покрокові інструкції, точки маршруту, орієнтовний час у дорозі та інші відповідні деталі, з нашої внутрішньої бази даних або зовнішніх API.

Система шаблонів Django дозволяє нам структурувати і форматувати інформацію про маршрут у візуально привабливий спосіб. За допомогою використання HTML, CSS та мови шаблонів Django можна створити зручний та функціональний інтерфейс, який представляє деталі маршруту в чіткій, зручній, зрозумілій та організованій формі. Шаблон може включє інтерактивні елементи, такі як карти, функцію масштабування та перемикання між різними видами карт, щоб покращити користувацький досвід.

Крім того, система автентифікації та авторизації користувачів Django дозволяє представникам транспортної компанії керувати доступом до управляння транспортними даними і надавати доступ до цієї функції лише відповідальним особам. Це дозволяє відділити управління транспортом від функцій доступних звичайним користувачам, таким як перегляд маршрутів, пошук маршрутів і інших.

Крім того, підтримка Django адаптивного веб-дизайну та мобільної оптимізації гарантує, що сторінка з маршрутом буде доступною і функціональною на різних пристроях і з різними розмірами екрану. Використовуючи адаптивні CSS фреймворки, такі як Bootstrap, ми можемо досягти адаптивного і візуально привабливого інтерфейсу користувача, який адаптується до пристрою користувача, будь то настільний комп'ютер, планшет або мобільний телефон.

\conclusions

Всебічне дослідження процесу розробки веб-застосунку для пошуку маршрутів, висвітлене в чотирьох розділах цього проекту, надало цінну інформацію про розробку надійного і зручного для користувачів додатку для пошуку маршрутів.

У першому розділі було розглянуто існуючі рішення, на прикладі популярних застосунків для пошуку маршрутів, підкресливши важливість ефективної і точної навігації в сучасному швидкоплинному світі. Проаналізувавши існуючі рішення, було отримано глибше розуміння викликів і вимог, а також бажань і портеб користувачів, пов'язаних з розробкою веб-додатку, який надає користувачам оптимальні маршрути.

У другому розділі було заглибилено в різні інструменти та мови програмування, доступні для розробки веб-додатків. Було досліджено сильні та слабкі сторони таких популярних технологій, як Python, JavaScript, Java та Go. Крім того, було проведено порівняння таких відомих фреймворків, як Django та Flask, які пропонують потужні можливості та спрощений досвід розробки для створення веб-застосунків.

Третій розділ присвячений тонкощам створення застосунку для пошуку маршрутів. Вона охоплює такі важливі аспекти, як дизайн адаптивного користувацького інтерфейсу та реалізацію основних функціональних можливостей. Використовуючи відповідні технології та фреймворки, такі як Django ORM, Bootstrap, було забезпечено ефективність, масштабованість та адаптивність застосунку до різних пристроїв.

Останній, п'ятий, розділ дозволив провести комплексну оцінку розробленого застосунку для пошуку маршрутів. За допомогою тестування та перевірки було ретельно проаналізовано функціональність, продуктивність та зручність використання застосунку.

Загалом, шлях від початкового огляду до огляду готового веб-застосунку підкреслив важливість ретельного планування, технологічної експертизи та підходу, орієнтованого на користувача. Завдяки використанню сучасних технологій веб-розробки та дотриманню найкращих практик було створено надійний та інтуїтивно зрозумілий додаток для пошуку маршрутів.

Таким чином, цей проект забезпечує комплексне дослідження розробки веб-застосунку для пошуку маршрутів. Розглянувши існуючі рішення, обравши відповідні технології, ретельно розробивши застосунок та провівши ретельні перевірки, було успішно створено цінний інструмент для невеликих транспортних компаній та користувачів, які шукають ефективну та надійну навігацію. Шлях від зародження до реалізації підкреслює важливість ретельного планування, порівняння різних веб-технологій та зосередження на створенні виняткового користувацького досвіду.