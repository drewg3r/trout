\vspace{-\baselineskip}

\subsection{MongoDB}
\label{subsec:mongoDB-subsection}

MongoDB - це база даних NoSQL, яка може бути підходящим вибором для сервісу пошуку маршрутів, пропонуючи унікальні функції, що задовольняють специфічні потреби такої системи в даних[17].

Як документно-орієнтована база даних, MongoDB забезпечує гнучке моделювання даних без використання схем. Ця гнучкість є перевагою при роботі з різноманітними та мінливими даними про маршрути та перевезення, оскільки дозволяє зберігати та отримувати складні вкладені структури без суворих обмежень схеми.

Однією з ключових особливостей MongoDB є здатність працювати з геопросторовими даними. Це робить її добре придатною для сервісів пошуку маршрутів, які вимагають ефективного запиту та аналізу інформації про місцезнаходження. MongoDB надає геопросторові індекси та широкий спектр геопросторових запитів, що дозволяє виконувати такі операції, як пошук маршрутів у межах певної відстані, запити на основі полігональних областей та обчислення близькості між точками.

Варто зазначити, що вибір MongoDB для сервісу пошуку маршрутів вимагає ретельного розгляду конкретних вимог і компромісів, пов'язаних з базою даних NoSQL. Хоча MongoDB пропонує переваги з точки зору гнучкості та масштабованості, вона може вимагати додаткових зусиль у моделюванні даних та розробці додатків, щоб забезпечити ефективну роботу запитів та узгодженість даних.

Таким чином, MongoDB - це гнучке і масштабоване рішення для зберігання і запиту маршрутних і транспортних даних у сервісі пошуку маршрутів. Її геопросторові можливості, можливості масштабування та підтримка спільноти роблять її життєздатним вибором для побудови ефективної та геопросторово орієнтованої системи, що відповідає вашим вимогам до пошуку маршрутів.