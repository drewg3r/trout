\subsection{Потреби індивідуальних користувачів}
\label{subsec:individual-users-needs}

Для більш дотального розуміння потреб індивідуальних користувачів слід 
розглянути групи людей, які користуються громадським транспортом:\\

	Люди, які їздять на роботу

    Люди, які їздять на роботу, є одними з основних користувачів громадського транспорту. Їм потрібна можливість швидко і легко знаходити найкоротші та найшвидші маршрути до місця призначення. Крім того, їм може знадобитися зберігати часто використовувані маршрути.\\

    Туристи

    Туристи є ще однією ключовою групою користувачів. Вони можуть бути не знайомі з системою громадського транспорту в новому місті і потребують можливості швидко і легко знаходити найкращі маршрути до популярних туристичних місць. Крім того, вони можуть захотіти зберегти свої маршрути для подальшого використання або поділитися ними з друзями та родиною.\\

    Студенти та школярі

    Студенти та школярі також значною мірою покладаються на громадський транспорт, щоб дістатися до школи чи університету. Їм потрібно вміти знаходити найшвидші та найкоротші маршрути до кампусу та додому. Їм також може знадобитися планувати свої маршрути з урахуванням розкладу занять або інших зобов'язань, таких як позакласні заходи або робота з неповним робочим днем.\\

	Ділові мандрівники

	Діловим мандрівникам часто потрібно дістатися до місця призначення якомога швидше. Вони повинні мати можливість знайти найшвидший і найефективніший маршрут, беручи до уваги трафік, розклад громадського транспорту та інші фактори.


Загалом, користувачі послуг маршрутизації громадського транспорту потребують точної та актуальної інформації про розклад та маршрути. Їм також потрібен зручний інтерфейс, який дозволяє швидко і легко знаходити потрібну інформацію. Іншими важливими функціями є можливість зберігати маршрути та ділитися ними, інформування в режимі реального часу про збої та затримки.