\subsection{Java}
\label{subsec:java-subsection}

Java - це широко використовувана мова програмування, відома своєю універсальністю, продуктивністю та надійністю. Вона особливо популярна у сфері корпоративної веб-розробки, де її об'єктно-орієнтована природа та розгалужена екосистема роблять її чудовим вибором для створення масштабованих та безпечних веб-додатків[8].

Основна перевага мови програмування Java полягає в її незалежності від платформи. Код Java компілюється у байт-код, який може працювати на будь-якій платформі за допомогою віртуальної машини Java (JVM). Це дозволяє розробникам писати код один раз і розгортати його на різних операційних системах, що робить Java ідеальним вибором для веб-додатків, які повинні бути сумісними з різними середовищами.

У контексті веб-розробки мова програмування Java зазвичай використовується для програмування на стороні сервера. Фреймворки на основі Java, такі як JavaServer Pages (JSP), JavaServer Faces (JSF) та Java Servlets, забезпечують основу для створення динамічних та інтерактивних веб-додатків. Ці фреймворки дозволяють розробникам обробляти HTTP-запити, генерувати динамічний веб-контент, керувати сеансами користувачів і взаємодіяти з базами даних.

Веб-додатки на Java часто використовують архітектуру Model-View-Controller (MVC), яка сприяє розділенню завдань і модульній розробці. Такі фреймворки, як Spring MVC та JavaServer Faces (JSF), забезпечують надійну реалізацію MVC, дозволяючи розробникам структурувати свої додатки та ефективно керувати складною бізнес-логікою.

Розгалужена екосистема Java також включає широкий спектр бібліотек та фреймворків, які покращують веб-розробку. Spring Framework, наприклад, надає комплексну підтримку для створення додатків корпоративного рівня, включаючи такі функції, як встановлення залежностей, доступ до даних та безпека. Інші фреймворки, такі як Hibernate, спрощують роботу з базами даних, а Apache Struts пропонує потужний фреймворк MVC для веб-додатків.

Крім того, мова програмування Java підтримує використання стандартної бібліотеки тегів JavaServer Pages (JSTL), яка надає набір тегів для поширених завдань веб-розробки, таких як ітерації, умови та форматування. Це полегшує розробникам створення динамічного контенту на веб-сторінках.

Крім того, мова програмування Java пропонує потужні функції безпеки та розвинений набір інструментів і бібліотек для захисту веб-додатків. Служба автентифікації та авторизації Java (JAAS) та фреймворки, такі як Spring Security, дозволяють розробникам впроваджувати автентифікацію, контроль доступу та інші заходи безпеки для захисту своїх веб-додатків.

Однак важливо зазначити, що веб-розробка на Java, як правило, передбачає більше налаштувань та конфігурацій порівняно з іншими мовами. Крива навчання може бути крутішою, особливо для початківців, через складність мови та різноманітність доступних фреймворків. Крім того, веб-додатки на Java можуть вимагати більше пам'яті та обчислювальної потужності порівняно з легшими альтернативами.

Загалом, надійність, незалежність від платформи та розгалужена екосистема мови програмування Java роблять її чудовим вибором для веб-розробки, особливо в корпоративному середовищі. Її продуктивність, функції безпеки та масштабованість роблять її добре придатною для створення великомасштабних, критично важливих веб-додатків.

Переваги Java:
\begin{itemize}
    \item Незалежність від платформи: код на Java працює на будь-якій платформі з віртуальною машиною Java (JVM), що забезпечує чудову портативність.
    \item Об'єктно-орієнтоване програмування: об'єктно-орієнтована природа Java дозволяє створювати модульний і багаторазовий код, що сприяє легкості супроводу і масштабованості.
    \item Надійність: сувора перевірка під час компіляції та обробка винятків у Java сприяють створенню стабільних та надійних додатків.
    \item Велика екосистема: Java має велику екосистему бібліотек, фреймворків та інструментів, які підтримують різні аспекти розробки, прискорюючи процес розробки.
    \item Безпека: Java надає вбудовані засоби безпеки, такі як пісочниця, перевірка байт-коду та криптографічні бібліотеки для розробки безпечних додатків.
\end{itemize}

Недоліки Java:
\begin{itemize}
    \item Крива навчання: Java має круту криву навчання, особливо для початківців, через складність синтаксису та необхідність розуміння об'єктно-орієнтованих принципів.
    \item Багатослівність: синтаксис Java може бути багатослівним, що вимагає більшої кількості рядків коду порівняно з іншими мовами для досягнення подібної функціональності.
    \item Споживання пам'яті: програми на Java зазвичай потребують більше пам'яті порівняно з легшими мовами, що може бути важливим фактором для середовищ з обмеженими ресурсами.
    \item Час запуску: програми на Java, як правило, мають довший час запуску порівняно з інтерпретованими мовами, оскільки вони потребують ініціалізації JVM та завантаження залежностей.
    \item Обмежений паралелізм: хоча Java забезпечує підтримку паралелізму за допомогою потоків і бібліотек, таких як java.util.concurrent, керування паралельним програмуванням може бути складним і схильним до помилок.
\end{itemize}