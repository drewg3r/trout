\subsection{JavaScript}
\label{subsec:js-subsection}

JavaScript - це універсальна і широко використовувана мова програмування, яка в першу чергу використовується для розробки інтерактивних і динамічних веб-додатків. Вона підтримується всіма основними веб-браузерами, що робить її фундаментальною технологією для інтерфейсної веб-розробки. JavaScript - це високорівнева, інтерпретована мова з простим і гнучким синтаксисом, яка дозволяє розробникам створювати багатий і цікавий користувацький досвід в Інтернеті[9].

Однією з ключових переваг мови програмування JavaScript є її здатність працювати безпосередньо в браузері, що дозволяє створювати сценарії на стороні клієнта. Це означає, що код JavaScript може виконуватися на пристрої користувача, зменшуючи потребу в обробці на стороні сервера і підвищуючи швидкість відгуку веб-додатків. Мова програмування JavaScript може маніпулювати елементами HTML, обробляти користувацькі події, виконувати перевірку вводу та динамічно оновлювати вміст і зовнішній вигляд веб-сторінок.

Окрім розробки інтерфейсів, мова програмування JavaScript також використовується для розробки бекенд-версій з появою серверних фреймворків JavaScript, таких як Node.js. Node.js дозволяє розробникам створювати масштабовані та ефективні веб-сервери та додатки за допомогою мови програмування JavaScript. За допомогою Node.js JavaScript можна використовувати для обробки серверної логіки, виконання операцій з базами даних та створення RESTful API.

Мова програмування JavaScript має велику екосистему бібліотек та фреймворків, які розширюють його можливості та спрощують завдання веб-розробки. Такі популярні бібліотеки, як React, Angular та Vue.js, надають потужні інструменти для створення складних користувацьких інтерфейсів та управління станом у великомасштабних додатках. Ці фреймворки підвищують продуктивність завдяки багаторазовому використанню компонентів, ефективному зв'язуванню даних та широкій підтримці спільноти.

Більше того, мова програмування JavaScript вийшов за межі веб-розробки і тепер використовується для розробки мобільних додатків (React Native, Ionic), десктопних додатків (Electron) і навіть машинного навчання (TensorFlow.js). Така універсальність дозволяє розробникам використовувати свої навички JavaScript на різних платформах і в різних сферах.

Незважаючи на свої сильні сторони, мова програмування JavaScript має деякі обмеження. Він може бути схильний до проблем сумісності з браузерами, оскільки різні браузери можуть дещо по-різному інтерпретувати JavaScript-код. Крім того, гнучкість мови програмування JavaScript може призвести до потенційних пасток, якщо використовувати його необережно, наприклад, проблеми зі змінними в області видимості та примусове використання типів. Однак, правильне кодування та використання сучасних функцій та інструментів мови програмування JavaScript може допомогти пом'якшити ці проблеми.

Загалом, JavaScript - це потужна і широко розповсюджена мова програмування, яка дозволяє розробникам створювати динамічні та інтерактивні веб-додатки. Її універсальність, велика екосистема та широка підтримка браузерами роблять її популярним вибором для фронтенд- та бекенд-веб-розробки, а також для інших сфер застосування.

Переваги:
\begin{itemize}
    \item Широка сумісність з браузерами: JavaScript підтримується всіма основними веб-браузерами, що забезпечує широку сумісність веб-додатків.
    \item Інтерактивна взаємодія з користувачем: JavaScript дозволяє    створювати динамічні та інтерактивні користувацькі інтерфейси, покращуючи загальний користувацький досвід.
    \item Широка екосистема: JavaScript має широку екосистему бібліотек, фреймворків та інструментів, які спрощують завдання розробки та прискорюють реалізацію проектів.
    \item Багаторазове використання: бібліотеки та фреймворки JavaScript надають багаторазові компоненти та модулі, що полегшує повторне використання коду та пришвидшує розробку.
\end{itemize}

Недоліки:
\begin{itemize}
    \item Сумісність з браузерами: різні браузери можуть по-різному інтерпретувати код JavaScript, що вимагає додаткових зусиль для тестування та налагодження кросбраузерної сумісності.
    \item Обмежена продуктивність: інтерпретована природа JavaScript може призвести до зниження продуктивності порівняно з мовами нижчого рівня, особливо для завдань з інтенсивними обчисленнями.
    \item Проблеми безпеки: як сценарії на стороні клієнта, код JavaScript доступний користувачам, що робить його вразливим до таких атак, як впровадження коду та міжсайтовий скриптинг, якщо він не захищений належним чином.
    \item Відсутність строгої типізації: JavaScript має динамічну типізацію, що може призвести до потенційних помилок і ускладнити підтримку коду.
    \item Крива навчання: JavaScript має криву навчання, особливо для розробників, які переходять з інших мов, через його унікальні особливості та асинхронну модель програмування.
\end{itemize}