\subsection{Go}
\label{subsec:go-subsection}

Go - це статично типізована, скомпільована мова програмування, розроблена компанією Google. Вона створена для того, щоб бути ефективною, лаконічною та добре масштабованою. Хоча Go є мовою загального призначення, вона набула популярності у спільноті веб-розробників завдяки своїй чудовій підтримці для створення високопродуктивних веб-додатків[10].

Простота і мінімалізм мови програмування Go роблять її привабливим вибором для веб-розробки. Її синтаксис чистий і легкий для читання, що дозволяє розробникам писати лаконічний і виразний код. Мова програмування Go робить акцент на читабельності коду, що полегшує його розуміння та підтримку великих кодових баз з часом.

Продуктивність мови програмування Go є визначною особливістю. Це скомпільована мова, яка створює власний машинний код, що дозволяє створювати високоефективні та швидкодіючі додатки. Вбудована в Go модель паралелізму, Goroutines, дозволяє легко та ефективно програмувати паралельно, що робить її добре придатною для обробки високого трафіку та паралельних запитів у веб-додатках.

Що стосується фреймворків для веб-розробки, мова програмування Go пропонує кілька популярних варіантів. Одним з найпоширеніших фреймворків є стандартний пакет бібліотек Go ``net/http''. Він надає надійний набір інструментів для створення веб-серверів, управління маршрутизацією, обслуговування статичних файлів та управління HTTP-запитами і відповідями. Простота і потужність стандартної бібліотеки дозволяють розробникам створювати веб-додатки без необхідності важких зовнішніх залежностей.

Ще одним помітним фреймворком в екосистемі мови програмування Go є Gin. Gin - це легкий і високопродуктивний веб-фреймворк, який надає додаткові функції та абстракції на додаток до стандартної бібліотеки. Він пропонує швидкий маршрутизатор, підтримку проміжного програмного забезпечення та різноманітні утиліти для спрощення типових завдань веб-розробки.

Сильний акцент мови програмування Go на простоті та ефективності поширюється і на його розгортання. Додатки на Go компілюються в автономні двійкові файли, що спрощує розгортання та усуває необхідність у додаткових залежностях під час виконання. Ця характеристика робить Go добре придатною для створення мікросервісів або розгортання додатків на хмарних платформах.

Хоча мова програмування Go пропонує численні переваги для веб-розробки, важливо враховувати її відносну незрілість у порівнянні з більш усталеними мовами. Екосистема Go все ще розвивається, і доступність сторонніх бібліотек та інструментів може бути більш обмеженою порівняно з такими мовами, як Python або JavaScript. Проте, основна мова та стандартна бібліотека є стабільними та добре підтримуються спільнотою Go.

Отже, Go - це потужна мова програмування для веб-розробки, яка пропонує простоту, ефективність та високу продуктивність. Її чистий синтаксис, вбудована модель паралелізму та всеосяжна стандартна бібліотека роблять її чудовим вибором для створення масштабованих та високопродуктивних веб-додатків. Завдяки зростаючій екосистемі та підтримці спільноти, Go продовжує набирати популярність як найкраща мова для проектів веб-розробки.

Переваги Go (Golang):
\begin{itemize}
    \item Продуктивність: компільована природа Go та ефективна модель паралелізму сприяють створенню високопродуктивних веб-додатків.
    \item Спрощеність: чистий синтаксис та мінімалістичний дизайн Go сприяють читабельності коду та легкості його супроводу.
    \item Конкурентність: підпрограми та вбудовані примітиви паралельності спрощують паралельне програмування, дозволяючи ефективно обробляти декілька запитів.
    \item Стандартна бібліотека: стандартна бібліотека Go надає основні інструменти для веб-розробки, що полегшує початок роботи без важких зовнішніх залежностей.
    \item Розгортання: програми на Go компілюються у автономні двійкові файли, що спрощує розгортання і усуває необхідність у додаткових залежностях під час виконання.
\end{itemize}

Недоліки Go (Golang):
\begin{itemize}
    \item Зрілість екосистеми: екосистема Go є відносно молодою, і доступність сторонніх бібліотек та інструментів може бути більш обмеженою у порівнянні з більш усталеними мовами.
    \item Крива навчання: Go має власний набір парадигм і концепцій програмування, які можуть потребувати певного вивчення і пристосування для розробників, які не знайомі з мовою.
    \item Невелика кількість бібліотек: хоча екосистема Go розвивається, кількість доступних бібліотек і фреймворків може бути меншою у порівнянні з більш популярними мовами, такими як Python або JavaScript.
    \item Відсутність узагальнень: наразі Go не підтримує узагальнення, що іноді може призвести до дублювання коду або більш багатослівних рішень.
    \item розмір спільноти: хоча спільнота Go є активною і доброзичливою, вона може бути меншою порівняно зі спільнотами більш поширених мов.
\end{itemize}