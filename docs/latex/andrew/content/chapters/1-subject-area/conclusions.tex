\uchapter{Висновки до розділу 1}

За останні роки сфера планування маршрутів і навігації зазнала значного прогресу, зумовленого зростаючим попитом на ефективні рішення з пошуку маршрутів. Існуючі рішення, такі як Google Maps, Moovit, Citymapper та Transit, стали популярним вибором, пропонуючи широкий спектр можливостей та функцій для задоволення різноманітних потреб користувачів. Ці платформи надають вичерпну інформацію про маршрути, оновлення в режимі реального часу та зручні інтерфейси, що покращує загальний досвід навігації.

Однак існують певні обмеження та проблеми, пов'язані з існуючими рішеннями, особливо в контексті невеликих міст або регіонів з менш розвиненою транспортною мережею. У таких районах доступність і точність маршрутної інформації може бути обмеженою, що заважає користувачам ефективно планувати свої подорожі.

Надаючи адміністраторам можливість керувати та оновлювати маршрути, додаток гарантує доступність актуальної та точної інформації. Тим часом користувачі можуть використовувати інтуїтивно зрозумілий інтерфейс для пошуку маршрутів, перегляду детальної інформації та ефективного планування своїх поїздок. Додаток має на меті спростити процес навігації, покращити доступ до транспортних можливостей та покращити загальний досвід подорожей для користувачів.

Завдяки орієнтації на потреби користувачів, точній інформації про маршрути та зручному інтерфейсу, додаток прагне стати цінним інструментом для приватних осіб, малих компаній, туристів та мешканців невеликих міст. Вирішуючи конкретні проблеми, з якими стикаються в цих сферах, і забезпечуючи надійну функціональність планування маршрутів, додаток має потенціал для того, щоб мати значний вплив і зарекомендувати себе як успішне рішення на ринку.

Насамкінець, завдяки своїм спеціальним функціям, таким як можливість додавання маршрутів власноруч, додаток надає можливість представникам невеликих транспортних компаній знаходити клієнтів для своїх послуг.