\subsection{Потреби людей, що їздять на роботу}
\label{subsec:workers-needs-subsection}

Люди, які щодня їздять на роботу, мають кілька потреб, коли мова йде про сервіс для планування маршруту:

Найшвидші та найефективніші маршрути: людям, які їздять на роботу щодня, потрыбно шукати найшвидші та найефективніші маршрути, щоб дістатися до місця призначення. Вони можуть хотіти уникнути заторів на дорогах і знайти найшвидший спосіб дістатися до роботи.

Кілька варіантів транспорту: Людям, які їздять на роботу, може знадобитися послуга, що надає їм кілька варіантів транспорту.

Зручний інтерфейс: Зручний інтерфейс важливий для пасажирів, які хочуть швидко і легко планувати свої щоденні поїздки. Їм може знадобитися простий у використанні сервіс, який надає чітку та стислу інформацію про варіанти маршрутів.

Недорогі маршрути: Користувачі також можуть шукати економічно ефективні рішення для своїх щоденних поїздок на роботу. Це означає, що користувачі нададуть перевагу робити якомога меньше пересадок. Користувачі також можуть обирати транспорт за власним вподобанням.

Оновлення в реальному часі: Пасажирам може знадобитися інформація в режимі реального часу про стан дорожнього руху та затримки на їхньому маршруті. Ця інформація може допомогти їм відповідно спланувати свою подорож і уникнути заторів.

Загалом, сервіс для планування маршрутів для людей, які щодня їздять на роботу, повинен бути ефективним, мати оновлення в режимі реального часута кілька видів транспорту, а також бути зручним у користуванні.