\subsection{Потреби невеликих транспортних компаній}
\label{subsec:companies-needs-subsection}

Малі компанії мають різні потреби, коли йдеться про планування маршрутів власним транспортом. Ось деякі з основних потреб:

Донесення власних послуг до користувачів: для транспортних компаній, особливо невеликих, важливим є інформуванні клієнтів про свої послуги. Так як транспортних компаній існує дуже багато, ненав'язливим способом показати клієнтам можливості своєї компанії є сервіс, з яким користуач сам знайде необхіжні для нього транспортні з'єднання та потрібну інформацію про них. Це є чудовим способом отримати нових клієнтів, не набридаючи рекламою.

Налаштовуваня та менеджмент маршрутів: невеликим компаніям потрібен сервіс, який дозволяє слідкувати за даними транспортних маршрутів, вносити зміни, бачити недоліки в створених маршрутах. Також завдяки цьому сервісу, можна побачити маршрути з боку користувача, прокладаючи маршрути між розними зупинками. Також компаніям може знадобитись можливість вносити тимчасові зміна в маршрути, наприклад закривати зупинку через проведення ремонтних робіт.

Зручний інтерфейс: малим компаніям потрібен сервіс, який простий у використанні і не вимагає великих технічних знань. Тому сервіс має бути інтуїтивно зрозумілим і зручним для користувача.

Ефективне прокладання маршрутів: малим компаніям потрібен сервіс, здатний ефективно прокладати маршрути. Це потрібно для того щоб користувач бачив всі можливості транспортної компанії.

Оновлення в режимі реального часу: малим компаніям потрібен сервіс, який надає оновлення в реальному часі про трафік, погоду та інші фактори, які можуть вплинути на час в дорозі.

Надійна користувацька підтримка: невеликим компаніям потрібна послуга, яка забезпечує надійну підтримку клієнтів у разі виникнення питань або проблем. У них може не вистачати ресурсів для самостійного усунення технічних проблем, тому їм потрібен сервіс, який пропонує оперативну та кваліфіковану підтримку.
