\subsection{Можливості застосунку}
\label{subsec:features-subsection}

В ході дослідження існуючих рішень було виділено декі ряд ключових функцій, які дозволяють користувачам ефективно та раціонально керувати своїми маршрутами. Список основних функцій, які повиння бути присутні в застосунку:\\

\begin{itemize}
    \item Додавання маршрутів: Функція додавання маршрутів дозволяє адміністраторам додавати нові маршрути до системи. Вони можуть вводити такі дані, як пункт відправлення, пункт призначення, проміжні зупинки та будь-яку іншу додаткову інформацію, що стосується маршруту. Вони також можуть надати інформацію про розклад, включаючи час відправлення та прибуття, частоту та будь-які зміни на основі робочих та вихідних днів.
    Інтерфейс адміністратора надає зручну форму або інтерфейс для введення всіх необхідних даних, забезпечуючи точність і повноту.\\

    \item Редагування маршрутів: Функція редагування маршруту дозволяє адміністраторам змінювати існуючі маршрути. Вони можуть оновлювати таку інформацію, як розклад, зупинки або будь-які інші важливі деталі. Адміністратори можуть вносити зміни в режимі реального часу, гарантуючи, що додаток відображає точну та актуальну інформацію про маршрут. Процес редагування може включати коригування карти маршруту, додавання або видалення зупинок, зміну розкладу або оновлення будь-якої іншої інформації, пов'язаної з маршрутом.\\

    \item Видалення маршрутів: Функція видалення маршрутів дозволяє адміністраторам видаляти маршрути, які більше не використовуються або стали застарілими. Коли маршрут видалено, він більше не буде видимим для користувачів, що запобігає доступу до застарілої або невірної інформації. Видалення маршрутів гарантує, що база даних додатку залишається впорядкованою та актуальною, надаючи користувачам надійні та актуальні варіанти маршрутів.\\

    \item Приховування маршрутів: Функція приховування маршрутів дозволяє адміністраторам не выдображати маршрути, які більше не використовуються або стали застарілими, при цьому не видаляючи їх.\\

    \item Пошук маршрутів: Функція пошуку маршрутів дозволяє користувачам шукати маршрути на основі бажаних початкових і кінцевих точок. Вони можуть вводити адреси, орієнтири або цікаві місця, щоб знайти найзручніші маршрути. Додаток використовує алгоритми для розрахунку оптимальних варіантів маршруту, враховуючи такі фактори, як відстань, приблизний час у дорозі та доступні види транспорту. Користувачі можуть обирати між різними видами транспорту, такими як автобуси, трамваї, поїзди та іншими видами транспорту.\\

    \item Перегляд деталей маршруту: Функція деталізації маршруту дозволяє користувачам переглядати інформацію про прокладений маршрут. Детальна інформація про кожну зупинку, включаючи час прибуття і відправлення, може бути надана, що дозволяє користувачам ефективно планувати свою подорож. Користувачі також можуть отримати доступ до додаткових даних, таких як приблизний час у дорозі та відстань.\\

\end{itemize}


Також серед існуючих додатків більшість існують перш за все як мобільний додаток, ще не підходить для всіх пристроїв. В свою чергу веб-додаток може відображатися як на телеффоні так і на комп'ютері чи інших присторх, які мають браузер.