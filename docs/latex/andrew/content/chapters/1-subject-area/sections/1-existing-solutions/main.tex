\section{Загальний огляд програм для прокладання маршрутів транспорту}
\label{sec:existing-sulutions}

Програми для пошуку найкоротших маршрутів широко використовуються в різних сферах. Ці програми використовуються для пошуку найкоротшого шляху між двома точками на основі різних критеріїв, таких як відстань, час, вартість і так далі. Деякі з найпопулярніших алгоритмів пошуку найкоротшого маршруту, що використовуються в цих програмах, включають алгоритм Дейкстри, алгоритм A*, алгоритм Беллмана-Форда та алгоритм пошуку в ширину.

Однією з популярних програм для пошуку найкоротших маршрутів громадського транспорту є Google Maps. Google Maps дозволяє користувачам вводити своє початкове місцезнаходження та пункт призначення, а потім пропонує кілька можливих маршрутів на вибір. Програма враховує поточне місцезнаходження користувача, а також дані про трафік і громадський транспорт у реальному часі, щоб запропонувати найефективніший маршрут[2].

Іншою програмою, яка спеціалізується на пошуку найкоротших маршрутів на громадському транспорті, є Moovit. Moovit пропонує інформацію про громадський транспорт в режимі реального часу, включаючи розклади руху автобусів та потягів, і надає користувачам найшвидші та найефективніші маршрути. Програма також дозволяє користувачам відслідковувати свій автобус чи потяг в режимі реального часу, надаючи точний час прибуття та відправлення[3].

Також досить популярною є Citymapper - це додаток для громадського транспорту, який пропонує інформацію про транспорт у реальному часі для більш ніж 80 міст по всьому світу. Він пропонує планування маршрутів, реальний час прибуття та відправлення, а також покрокові інструкції для систем громадського транспорту, таких як автобуси, поїзди та метро. Citymapper також надає інформацію про ціни на проїзд і пропонує функцію ``дощової'' маршрутизації, яка пропонує маршрути, що мінімізують час, проведений на вулиці в погану погоду[4].

Transit --- це додаток для громадського транспорту, який пропонує інформацію про громадський транспорт у реальному часі для понад 200 міст по всьому світу. Він пропонує планування маршрутів, реальний час прибуття та відправлення, а також покрокові вказівки для систем громадського транспорту, таких як автобуси, поїзди та метро. Transit також надає інформацію про системи спільного користування велосипедами та пропонує інформацію в режимі реального часу про наявність велосипедів та місць для паркування[5].

Загалом, згадані вище програми пропонують користувачам різні варіанти пошуку найкоротших маршрутів на громадському транспорті. Кожна програма має свої унікальні особливості та переваги, тому вибір, якою з них користуватися, залежатиме від потреб та вподобань користувача.


Нижче кожну з програм буде розглянуто більш детально.

% Subsections
\subsection{Google Maps}
\label{subsec:google-maps-subsection}

Google Maps -- це популярний сервіс, який пропонує широкий спектр функцій, включаючи оновлення дорожнього руху в режимі реального часу, супутникові знімки, а також вказівки щодо проїзду, пішохідних маршрутів і маршрутів громадського транспорту. Зручний інтерфейс та безперешкодна інтеграція з іншими сервісами Google роблять його популярним серед користувачів по всьому світу.

Однією з ключових переваг Google Maps є великий набір даних, який використовується для надання користувачам точної та актуальної інформації про місцезнаходження та маршрути. Сервіс використовує поєднання супутникових знімків, геопросторових даних і створеного користувачем контенту, щоб надати комплексне уявлення про місцезнаходження та навігаційні маршрути.

З точки зору користувацького досвіду, Google Maps пропонують інтуїтивно зрозумілий інтерфейс, який є простим у використанні та навігації. Користувачі можуть ввести бажане місце розташування і отримати докладні вказівки, що включають покрокові інструкції, приблизний час у дорозі та альтернативні маршрути. Сервіс також дозволяє користувачам налаштовувати вигляд карти та шукати певні об'єкти, такі як ресторани, магазини та визначні пам'ятки.

Google Maps доступний в Україні і надає достовірну інформацію про маршрути та напрямки у великих містах, таких як Київ, Харків та Львів. Однак точність і доступність інформації може відрізнятися в менших містах і сільській місцевості, оскільки Google Maps значною мірою покладається на контент, створений користувачами, який доповнює набір даних.

Загалом, Google Maps -- це потужний і зручний сервіс, який пропонує широкий спектр функцій для навігації як у великих, так і в малих містах. Хоча він може не забезпечувати повне покриття всіх областей, він залишається цінним ресурсом для користувачів, які шукають надійні транспортні сполучення та інформацію про місцезнаходження.


\subsection{Moovit}
\label{subsec:moovit-subsection}

Moovit -- популярний мобільний додаток, який в режимі реального часу інформує про автобусні, залізничні, метрополітен та інші громадські транспортні засоби в більш ніж 3 000 міст по всьому світу, включаючи Україну. Додаток покликаний полегшити користувачам навігацію громадським транспортом, надаючи їм точну та актуальну інформацію про розклади, маршрути та затримки[3].

Moovit пропонує широкий спектр можливостей, включаючи дані про транзит в режимі реального часу, планування поїздок та навігацію в реальному часі. Користувачі можуть встановлювати свої домашні та робочі місця і отримувати сповіщення про будь-які затримки або зміни в їхніх варіантах пересування.

Користувацький інтерфейc Moovit інтуїтивно зрозумілий і простий у використанні. Користувачі можуть шукати варіанти пересування, вводячи своє поточне місцезнаходження та пункт призначення, і додаток надає список варіантів пересування та приблизний час у дорозі. Користувачі також можуть переглядати карти та розклади транспорту і зберігати свої улюблені маршрути для подальшого використання.

Moovit отримує дані від різних транспортних агентств та операторів, а також краудсорсингові дані від користувачів. Дані додатку охоплюють широкий спектр варіантів пересування, включаючи автобуси, поїзди, метро та легку залізницю.

Moovit добре працює в Україні, особливо у великих містах, де є більш розвинена інфраструктура громадського транспорту. Однак додаток може бути менш корисним у менших містах та сільській місцевості, де можливості громадського транспорту обмежені. Користувачі також повинні знати, що точність даних додатку в режимі реального часу може змінюватися в залежності від якості даних, наданих місцевими транспортними агентствами.

Загалом, Moovit є корисним інструментом для всіх, хто покладається на громадський транспорт для пересування, надаючи інформацію в режимі реального часу і полегшуючи планування поїздок. Однак, користувачі в невеликих містах або сільській місцевості можуть знайти додаток менш корисним через обмежену доступність громадського транспорту.


\subsection{Citymapper}
\label{subsec:citymapper-subsection}

Citymapper -- популярний транспортний додаток, який надає користувачам інформацію в режимі реального часу та варіанти планування маршрутів для різних видів транспорту. Він доступний у понад 80 містах світу.

Однією з головних переваг Citymapper є його зручний інтерфейс, який дозволяє користувачам легко вводити пункт призначення та переглядати різні варіанти маршрутів, орієнтовний час у дорозі, а також оновлення в режимі реального часу про затримки та перебої в роботі транспорту. Додаток також пропонує персоналізовані функції, такі як можливість зберігати часто використовувані місця та маршрути, а також кнопку "Go", яка швидко розраховує найшвидший маршрут на основі поточного місцезнаходження користувача.

Citymapper має доступ до широкого спектру транспортних даних, включаючи дані про транзит у реальному часі від місцевих транспортних служб, дані про поїздки на попутках від таких провайдерів, як Uber і Lyft, та дані про велопрокат від таких провайдерів, як Lime і JUMP. Це дозволяє додатку надавати точну та актуальну інформацію про час у дорозі, розклад руху транспорту та іншу важливу транспортну інформацію.

З точки зору покриття в Україні, Citymapper наразі доступний лише в Києві. Однак широке покриття додатку в інших великих містах світу свідчить про те, що він має потенціал для розширення в інших містах України та надання цінної транспортної інформації користувачам у менших містах.

Загалом, Citymapper -- це потужний та зручний транспортний додаток, який пропонує широкий спектр функцій та доступ до різноманітних наборів транспортних даних. Його доступність у Києві та інших великих містах світу робить його корисним інструментом як для мандрівників, так і для тих, хто користується громадським транспортом.

\subsection{Transit}
\label{subsec:transit-subsection}

Transit -- це популярний мобільний додаток, який надає користувачам інформацію про громадський транспорт у режимі реального часу. Додаток доступний на платформах iOS та Android і пропонує безліч функцій, які допомагають користувачам орієнтуватися в системах громадського транспорту в різних містах світу[5].

Однією з головних переваг Transit є його зручний інтерфейс. Додаток розроблений так, щоб бути інтуїтивно зрозумілим і простим у використанні, з чіткими і стислими інструкціями для доступу до різних функцій. Користувачі можуть легко шукати транзитні маршрути та отримувати оновлення в режимі реального часу про стан обраних ними маршрутів, включаючи затримки та збої в роботі.

Для надання користувачам точної та актуальної інформації Transit покладається на низку різних джерел даних. Окрім офіційних даних транспортних агентств, додаток також інтегрує дані зі сторонніх джерел, таких як краудсорсингова інформація та дані зі служб спільного користування велосипедами.

Що стосується доступності та покриття в Україні, то наразі Transit доступний лише в обмеженій кількості міст. Однак, додаток надає інформацію для великих українських міст - Києва, Львова та Одеси, а також кількох інших менших міст.

Однією з унікальних особливостей Transit є його інтеграція з іншими транспортними опціями, такими як сервіси спільного користування автомобілями та системами прокату велосипедів. Це дозволяє користувачам легко планувати і здійснювати мультимодальні поїздки з використанням різних видів транспорту.

Загалом, Transit є корисним і зручним інструментом для користувачів, які хочуть орієнтуватися в системах громадського транспорту в різних містах. Хоча його покриття в Україні наразі обмежене, додаток пропонує точну та достовірну інформацію для найбільших міст країни.


\subsection{Порівняння існуючих рішень}
\label{subsec:comparison-subsection}

\begin{centering}
\resizebox{16cm}{!}{%
\begin{tabular}{|l| p{3.5cm} | p{3.6cm} | p{2.9cm} | p{3cm} |}
\hline
\textbf{Застосунок} & \textbf{Можливості} & \textbf{Користувацький досвід} & \textbf{Набори даних} & \textbf{Доступність в Україні} \\ \hline
	Google Maps &
 	Планування маршрутів, оновлення маршрутів у реальному часі &
 	Простий у використанні, інтуїтивно зрозумілий інтерфейс &
 	Глобальне охоплення, включає дані від місцевих транспортних компаній &
 	Доступний лише у великих містах України \\ \hline
	
	Moovit &
	Планування маршрутів, оновлення маршрутів у реальному часі &
	Чіткий та зрозумілий інтерфейс &
	Глобальне покриття, включає дані від місцевих транспортних компаній &
	Доступний лише у великих містах України \\ \hline

	Citymapper &
	Планування маршрутів, оновлення маршрутів у реальному часі &
	Цікавий, містить інформацію про погоду та розваги & 
	Обмежене покриття за межами великих міст &
	Не доступний в Україні \\ \hline

	Transit &
	Планування маршрутів, оновлення маршрутів у реальному часі &
	Простий і зрозумілий &
	Обмежене покриття за межами великих міст &
	Доступний лише у великих містах України \\ \hline
\end{tabular}
}
\end{centering}
