\subsection{Moovit}
\label{subsec:moovit-subsection}

Moovit -- популярний мобільний додаток, який в режимі реального часу інформує про автобусні, залізничні, метрополітен та інші громадські транспортні засоби в більш ніж 3 000 міст по всьому світу, включаючи Україну. Додаток покликаний полегшити користувачам навігацію громадським транспортом, надаючи їм точну та актуальну інформацію про розклади, маршрути та затримки[3].

Moovit пропонує широкий спектр можливостей, включаючи дані про транзит в режимі реального часу, планування поїздок та навігацію в реальному часі. Користувачі можуть встановлювати свої домашні та робочі місця і отримувати сповіщення про будь-які затримки або зміни в їхніх варіантах пересування.

Користувацький інтерфейc Moovit інтуїтивно зрозумілий і простий у використанні. Користувачі можуть шукати варіанти пересування, вводячи своє поточне місцезнаходження та пункт призначення, і додаток надає список варіантів пересування та приблизний час у дорозі. Користувачі також можуть переглядати карти та розклади транспорту і зберігати свої улюблені маршрути для подальшого використання.

Moovit отримує дані від різних транспортних агентств та операторів, а також краудсорсингові дані від користувачів. Дані додатку охоплюють широкий спектр варіантів пересування, включаючи автобуси, поїзди, метро та легку залізницю.

Moovit добре працює в Україні, особливо у великих містах, де є більш розвинена інфраструктура громадського транспорту. Однак додаток може бути менш корисним у менших містах та сільській місцевості, де можливості громадського транспорту обмежені. Користувачі також повинні знати, що точність даних додатку в режимі реального часу може змінюватися в залежності від якості даних, наданих місцевими транспортними агентствами.

Загалом, Moovit є корисним інструментом для всіх, хто покладається на громадський транспорт для пересування, надаючи інформацію в режимі реального часу і полегшуючи планування поїздок. Однак, користувачі в невеликих містах або сільській місцевості можуть знайти додаток менш корисним через обмежену доступність громадського транспорту.
