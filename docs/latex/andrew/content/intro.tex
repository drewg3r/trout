\intro

У сучасному швидкоплинному світі час є цінним ресурсом. Поїздки з одного місця в інше можуть забирати багато часу і виснажувати, особливо якщо ви не знайомі з системою громадського транспорту. Громадський транспорт відіграє важливу роль у сучасному міському суспільстві, забезпечуючи економічно ефективний та екологічний спосіб пересування. Однак пошук найефективнішого маршруту може бути складним завданням, особливо у великих містах зі складною транспортною системою. У цьому контексті розробка веб-додатку, який може знаходити найкоротші та найшвидші маршрути громадського транспорту на основі даних розкладу, може значно покращити життя пасажирів.


Цей бакалаврський проект спрямований на розробку веб-додатку, який буде знаходити найкоротші та найшвидші маршрути громадського транспорту на основі даних розкладу. У проекті будуть використані Python і Django, дві популярні технології розробки веб-додатків, з акцентом на зручність користування та адаптивний дизайн.

Загалом, метою проекту є розробка рішення проблеми пошуку ефективних маршрутів громадського транспорту. Таким чином, проект може допомогти користувачам заощадити час і гроші, а також популяризувати використання громадського транспорту. В даній пояснювальній записці буде представлено методологію проекту, включаючи процес розробки програмного забезпечення та використані технології. Також будуть обговорені проблеми, що виникли під час реалізації проекту, та шляхи їх вирішення.
