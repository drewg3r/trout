\chapter{Дослідження та аналіз розробленого додатку}
\label{chap:development}

Розділ присвячений тестуванню, оцінюванню та аналізу розробленого додатку для пошуку маршрутів. Він охоплює різні аспекти забезпечення якості, гарантуючи надійність, функціональність і зручність використання додатку.

Тестування відіграє важливу роль у виявленні та вирішенні будь-яких потенційних проблем або помилок у додатку. Для перевірки точності та узгодженості функціональності пошуку маршрутів, обробки даних і загальної продуктивності системи застосовуються ретельні процедури тестування.

Крім того, цей розділ містить інструкцію користувача, з поясненням як використовувати розроблений додаток. Чітка та вичерпна документація допомагає користувачам зрозуміти особливості програми, її функціональні можливості та інструкції з використання. Вона слугує цінним ресурсом як для користувачів, так і для розробників, полегшуючи безпроблемне впровадження та ефективне усунення несправностей.

Загалом, розділ підкреслює важливість суворих практик забезпечення якості для створення надійного та зручного додатку для пошуку маршрутів. Завдяки всебічному тестуванню, ретельній перевірці та зворотному зв'язку з користувачами додаток може досягти оптимальної продуктивності, точності та задоволеності користувачів.


Отже, для того щоб знайти найкоротший маршрут, потрібно перейти на сторінку пошуку. На цій сторінці користувач може ввести переметри пошуку, такі як початкова та кінцева зупинки, а також бажаний час відправлення. Після введення параметрів пошуку, можна розпочати пошук маршруту, який відповідає заданим параметрам. Для цього потрібно натиснути кнопку ``Знайти маршрут''.

\begin{figure}[!htp]
	\centering
	\includegraphics[scale=0.3]{content/chapters/4-results/assets/img/form_unfilled.png}
	\caption{Форма для пошуку маршруту}
	\label{fig:form-unfilled}
\end{figure}

\newpage

Після того, як кнопку пошуку було натиснуто, система оброблює параметри та застосовує алгоритми, описані вище, для пошуку найкоротшого шляху. Після того, як маршрут було знайдено, користувач отримує сторінку зі знайденим маршрутом.

Приклад пошуку маршруту: пошук маршруту зі станції ``Центральний вокзал'' в Києві до станції ``Центральний автовокзал'' в Житомирі. В якості часу для пошуку обрано 12.05.2023 16:01. В результаті отримано маршрут, який включає 1 пересадку та поїздка займає 3 години 29 хвилин.

\begin{figure}[!htp]
	\centering
	\includegraphics[scale=0.3]{content/chapters/4-results/assets/img/form_filled.png}
	\caption{Приклад пошуку маршруту}
	\label{fig:form1}
\end{figure}

\begin{figure}[!htp]
	\centering
	\includegraphics[scale=0.3]{content/chapters/4-results/assets/img/route1.png}
	\caption{Приклад знайденого маршруту}
	\label{fig:route1}
\end{figure}

\newpage

Приклад пошуку маршруту: пошук маршруту зі станції ``ст. м. Іподром'' в Києві до станції ``Центральний вокзал'' в Вінниці. В якості часу відправлення для пошуку обрано 24.05.2023 9:50. В результаті отримано маршрут, який включає 3 пересадки та поїздка займає 2 години 31 хвилин.

\begin{figure}[!htp]
	\centering
	\includegraphics[scale=0.3]{content/chapters/4-results/assets/img/form2.png}
	\caption{Приклад пошуку маршруту}
	\label{fig:form2}
\end{figure}

\begin{figure}[!htp]
	\centering
	\includegraphics[scale=0.3]{content/chapters/4-results/assets/img/route2.png}
	\caption{Приклад знайденого маршруту}
	\label{fig:route2}
\end{figure}

\newpage

Приклад пошуку маршруту: пошук маршруту зі станції ``Центральний автовокзал'' в Києві до станції ``Центральний автовокзал'' в Житомирі. В якості часу для пошуку обрано 24.05.2023 09:00. В результаті отримано маршрут, який не має пересадок та поїздка займає 2 години 26 хвилин.

\begin{figure}[!htp]
	\centering
	\includegraphics[scale=0.3]{content/chapters/4-results/assets/img/form3.png}
	\caption{Приклад пошуку маршруту}
	\label{fig:form3}
\end{figure}

\begin{figure}[!htp]
	\centering
	\includegraphics[scale=0.3]{content/chapters/4-results/assets/img/route3.png}
	\caption{Приклад знайденого маршруту}
	\label{fig:route3}
\end{figure}




Приклад пошуку маршруту: пошук маршруту зі станції ``Центральний автовокзал'' в Києві до станції ``Центральний автовокзал'' в Житомирі. В якості часу для пошуку обрано 24.05.2023 09:00. В результаті отримано маршрут, який не має пересадок та поїздка займає 2 години 26 хвилин.

\begin{figure}[!htp]
	\centering
	\includegraphics[scale=0.3]{content/chapters/4-results/assets/img/form3.png}
	\caption{Приклад пошуку маршруту}
	\label{fig:form3}
\end{figure}

\begin{figure}[!htp]
	\centering
	\includegraphics[scale=0.3]{content/chapters/4-results/assets/img/route3.png}
	\caption{Приклад знайденого маршруту}
	\label{fig:route3}
\end{figure}

Приклад пошуку маршруту: пошук маршруту зі станції ``Центральний автовокзал'' в Києві до станції ``Центральний автовокзал'' в Житомирі. В якості часу для пошуку обрано 24.05.2023 09:00. В результаті отримано маршрут, який не має пересадок та поїздка займає 2 години 26 хвилин.

\begin{figure}[!htp]
	\centering
	\includegraphics[scale=0.3]{content/chapters/4-results/assets/img/form3.png}
	\caption{Приклад пошуку маршруту}
	\label{fig:form3}
\end{figure}

\begin{figure}[!htp]
	\centering
	\includegraphics[scale=0.3]{content/chapters/4-results/assets/img/route3.png}
	\caption{Приклад знайденого маршруту}
	\label{fig:route3}
\end{figure}

\uchapter{Висновки до розділу 3}

У цьому розділі було розглянуто різні аспекти розробки сервісу планування маршрутів. Спочатку було досліджено ключові функціональні можливості сервісу для пошуку маршрутів транспорту, включаючи пошук маршрутів, відображення маршрутів, додавання, обробку та відображення транспортних маршрутів. Функціонал пошуку маршрутів дозволяє користувачам знаходити найоптимальніші маршрути відповідно до їхніх уподобань та вимог. Функціонал відображення маршрутів представляє знайдені маршрути у зрозумілій та інформативній формі, надаючи користувачам покрокові інструкції, приблизний час у дорозі та будь-які важливі деталі.

Крім того, було обговорено, як Python та Django використовуються для реалізації цих функцій. Надійний фреймворк Django та розгалужена екосистема забезпечують ефективну та безпечну обробку даних, плавну інтеграцію з базами даних та безперешкодну взаємодію з інтерфейсними компонентами. Поєднання Python та Django забезпечує міцну основу для розробки надійного та масштабованого сервісу планування маршрутів.

Інтегруючи всі ці компоненти та функціональні можливості, розроблений сервіс планування маршрутів дозволяє користувачам легко шукати маршрути, переглядати детальну інформацію та приймати обґрунтовані рішення щодо своїх транспортних потреб. Продуманий дизайн, зручний інтерфейс та ефективна внутрішня реалізація працюють разом, щоб забезпечити комплексний та цінний досвід для користувачів.

У розділі також підкреслюється важливість користувацького досвіду та дизайну інтерфейсу. Завдяки ефективному дизайну інтерфейсу, зокрема адаптивному макету, інтуїтивно зрозумілій навігації та візуально привабливим елементам, користувачі можуть легко взаємодіяти з сервісом і отримувати доступ до потрібних функцій. Реалізація таких функцій, як пошук маршрутів, відображення маршрутів та взаємодія з транспортними даними, забезпечує безперебійну роботу користувачів.

Підсумовуючи, у розділі було розглянуто принципи проектування, функціональні можливості та методи реалізації, необхідні для розробки надійного та орієнтованого на користувача сервісу планування маршрутів. Застосовуючи ці принципи та використовуючи можливості Python і Django, сервіс для пошуку маршрутів має на меті покращити досвід користувачів у плануванні поїздок та оптимізувати їхні транспортні рішення.
