\chapter{Дослідження та аналіз розробленого додатку}
\label{chap:development}

Розділ присвячений тестуванню, оцінюванню та аналізу розробленого додатку для пошуку маршрутів. Він охоплює різні аспекти забезпечення якості, гарантуючи надійність, функціональність і зручність використання додатку.

Тестування відіграє важливу роль у виявленні та вирішенні будь-яких потенційних проблем або помилок у додатку. Для перевірки точності та узгодженості функціональності пошуку маршрутів, обробки даних і загальної продуктивності системи застосовуються ретельні процедури тестування.

Крім того, цей розділ містить інструкцію користувача, з поясненням як використовувати розроблений додаток. Чітка та вичерпна документація допомагає користувачам зрозуміти особливості програми, її функціональні можливості та інструкції з використання. Вона слугує цінним ресурсом як для користувачів, так і для розробників, полегшуючи безпроблемне впровадження та ефективне усунення несправностей.

Загалом, розділ підкреслює важливість суворих практик забезпечення якості для створення надійного та зручного додатку для пошуку маршрутів. Завдяки всебічному тестуванню, ретельній перевірці та зворотному зв'язку з користувачами додаток може досягти оптимальної продуктивності, точності та задоволеності користувачів.


Отже, для того щоб знайти найкоротший маршрут, потрібно перейти на сторінку пошуку. На цій сторінці користувач може ввести переметри пошуку, такі як початкова та кінцева зупинки, а також бажаний час відправлення. Після введення параметрів пошуку, можна розпочати пошук маршруту, який відповідає заданим параметрам. Для цього потрібно натиснути кнопку ``Знайти маршрут''.

\begin{figure}[!htp]
	\centering
	\includegraphics[scale=0.3]{content/chapters/4-results/assets/img/form_unfilled.png}
	\caption{Форма для пошуку маршруту}
	\label{fig:form-unfilled}
\end{figure}

\newpage

Після того, як кнопку пошуку було натиснуто, система оброблює параметри та застосовує алгоритми, описані вище, для пошуку найкоротшого шляху. Після того, як маршрут було знайдено, користувач отримує сторінку зі знайденим маршрутом.

Приклад пошуку маршруту: пошук маршруту зі станції ``Центральний вокзал'' в Києві до станції ``Центральний автовокзал'' в Житомирі. В якості часу для пошуку обрано 12.05.2023 16:01. В результаті отримано маршрут, який включає 1 пересадку та поїздка займає 3 години 29 хвилин.

\begin{figure}[!htp]
	\centering
	\includegraphics[scale=0.3]{content/chapters/4-results/assets/img/form_filled.png}
	\caption{Приклад пошуку маршруту}
	\label{fig:form1}
\end{figure}

\begin{figure}[!htp]
	\centering
	\includegraphics[scale=0.3]{content/chapters/4-results/assets/img/route1.png}
	\caption{Приклад знайденого маршруту}
	\label{fig:route1}
\end{figure}

\newpage

Приклад пошуку маршруту: пошук маршруту зі станції ``ст. м. Іподром'' в Києві до станції ``Центральний вокзал'' в Вінниці. В якості часу відправлення для пошуку обрано 24.05.2023 9:50. В результаті отримано маршрут, який включає 3 пересадки та поїздка займає 2 години 31 хвилин.

\begin{figure}[!htp]
	\centering
	\includegraphics[scale=0.3]{content/chapters/4-results/assets/img/form2.png}
	\caption{Приклад пошуку маршруту}
	\label{fig:form2}
\end{figure}

\begin{figure}[!htp]
	\centering
	\includegraphics[scale=0.3]{content/chapters/4-results/assets/img/route2.png}
	\caption{Приклад знайденого маршруту}
	\label{fig:route2}
\end{figure}

\newpage

Приклад пошуку маршруту: пошук маршруту зі станції ``Центральний автовокзал'' в Києві до станції ``Центральний автовокзал'' в Житомирі. В якості часу для пошуку обрано 24.05.2023 09:00. В результаті отримано маршрут, який не має пересадок та поїздка займає 2 години 26 хвилин.

\begin{figure}[!htp]
	\centering
	\includegraphics[scale=0.3]{content/chapters/4-results/assets/img/form3.png}
	\caption{Приклад пошуку маршруту}
	\label{fig:form3}
\end{figure}

\begin{figure}[!htp]
	\centering
	\includegraphics[scale=0.3]{content/chapters/4-results/assets/img/route3.png}
	\caption{Приклад знайденого маршруту}
	\label{fig:route3}
\end{figure}




Приклад пошуку маршруту: пошук маршруту зі станції ``Центральний автовокзал'' в Києві до станції ``Центральний автовокзал'' в Житомирі. В якості часу для пошуку обрано 24.05.2023 09:00. В результаті отримано маршрут, який не має пересадок та поїздка займає 2 години 26 хвилин.

\begin{figure}[!htp]
	\centering
	\includegraphics[scale=0.3]{content/chapters/4-results/assets/img/form3.png}
	\caption{Приклад пошуку маршруту}
	\label{fig:form3}
\end{figure}

\begin{figure}[!htp]
	\centering
	\includegraphics[scale=0.3]{content/chapters/4-results/assets/img/route3.png}
	\caption{Приклад знайденого маршруту}
	\label{fig:route3}
\end{figure}

Приклад пошуку маршруту: пошук маршруту зі станції ``Центральний автовокзал'' в Києві до станції ``Центральний автовокзал'' в Житомирі. В якості часу для пошуку обрано 24.05.2023 09:00. В результаті отримано маршрут, який не має пересадок та поїздка займає 2 години 26 хвилин.

\begin{figure}[!htp]
	\centering
	\includegraphics[scale=0.3]{content/chapters/4-results/assets/img/form3.png}
	\caption{Приклад пошуку маршруту}
	\label{fig:form3}
\end{figure}

\begin{figure}[!htp]
	\centering
	\includegraphics[scale=0.3]{content/chapters/4-results/assets/img/route3.png}
	\caption{Приклад знайденого маршруту}
	\label{fig:route3}
\end{figure}

\conclusions

Всебічне дослідження процесу розробки веб-застосунку для пошуку маршрутів, висвітлене в чотирьох розділах цього проекту, надало цінну інформацію про розробку надійного і зручного для користувачів додатку для пошуку маршрутів.

У першому розділі було розглянуто існуючі рішення, на прикладі популярних застосунків для пошуку маршрутів, підкресливши важливість ефективної і точної навігації в сучасному швидкоплинному світі. Проаналізувавши існуючі рішення, було отримано глибше розуміння викликів і вимог, а також бажань і портеб користувачів, пов'язаних з розробкою веб-додатку, який надає користувачам оптимальні маршрути.

У другому розділі було заглибилено в різні інструменти та мови програмування, доступні для розробки веб-додатків. Було досліджено сильні та слабкі сторони таких популярних технологій, як Python, JavaScript, Java та Go. Крім того, було проведено порівняння таких відомих фреймворків, як Django та Flask, які пропонують потужні можливості та спрощений досвід розробки для створення веб-застосунків.

Третій розділ присвячений тонкощам створення застосунку для пошуку маршрутів. Вона охоплює такі важливі аспекти, як дизайн адаптивного користувацького інтерфейсу та реалізацію основних функціональних можливостей. Використовуючи відповідні технології та фреймворки, такі як Django ORM, Bootstrap, було забезпечено ефективність, масштабованість та адаптивність застосунку до різних пристроїв.

Останній, п'ятий, розділ дозволив провести комплексну оцінку розробленого застосунку для пошуку маршрутів. За допомогою тестування та перевірки було ретельно проаналізовано функціональність, продуктивність та зручність використання застосунку.

Загалом, шлях від початкового огляду до огляду готового веб-застосунку підкреслив важливість ретельного планування, технологічної експертизи та підходу, орієнтованого на користувача. Завдяки використанню сучасних технологій веб-розробки та дотриманню найкращих практик було створено надійний та інтуїтивно зрозумілий додаток для пошуку маршрутів.

Таким чином, цей проект забезпечує комплексне дослідження розробки веб-застосунку для пошуку маршрутів. Розглянувши існуючі рішення, обравши відповідні технології, ретельно розробивши застосунок та провівши ретельні перевірки, було успішно створено цінний інструмент для невеликих транспортних компаній та користувачів, які шукають ефективну та надійну навігацію. Шлях від зародження до реалізації підкреслює важливість ретельного планування, порівняння різних веб-технологій та зосередження на створенні виняткового користувацького досвіду.