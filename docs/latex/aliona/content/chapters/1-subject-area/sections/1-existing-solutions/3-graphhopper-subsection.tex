\subsection{GraphHopper}
\label{subsec:graphhopper-subsection}

GraphHopper --- це потужний механізм маршрутизації, який пропонує 
гнучкі рішення для оптимізації маршрутів для різних сценаріїв 
використання. Програмне забезпечення можна використовувати для 
оптимізації маршрутів для автомобілів, велосипедів, пішоходів та 
громадського транспорту, що робить його універсальним рішенням для 
планування перевезень та оптимізації логістики.

Однією з ключових особливостей GraphHopper є його гнучкість. Програмне 
забезпечення можна легко налаштувати відповідно до конкретних потреб 
окремих підприємств чи організацій. Він пропонує цілий ряд варіантів 
конфігурації, які дозволяють користувачам налаштовувати алгоритми 
маршрутизації, вагові коефіцієнти та інші параметри відповідно до 
своїх потреб. Це робить його ідеальним рішенням для компаній, які 
потребують висококастомізованих рішень для маршрутизації, таких як 
компанії, що займаються доставкою або логістикою.

Ще однією сильною стороною GraphHopper є його масштабованість. 
Програмне забезпечення призначене для обробки великих обсягів даних і 
може легко обробляти запити на маршрутизацію для тисяч транспортних 
засобів або пунктів доставки. Це робить його чудовим вибором для 
компаній, які потребують оптимізації складних логістичних мереж або 
великих автопарків.

З точки зору користувацького досвіду, GraphHopper пропонує ряд API і 
веб-інтерфейсів, які дозволяють легко інтегрувати рішення для 
маршрутизації в існуюче програмне забезпечення або веб-сайти. 
Програмне забезпечення також добре задокументоване, з великою онлайн 
документацією та ресурсами підтримки, доступними для користувачів.

В Україні GraphHopper не так широко використовується, як деякі інші 
механізми маршрутизації, але він все ще є популярним вибором для 
підприємств та організацій, які потребують висококастомізованих рішень 
для маршрутизації. Зазвичай це програмне забезпечення використовується 
великими компаніями або організаціями з більш складними потребами в 
маршрутизації, такими як логістичні провайдери, транспортні компанії 
та муніципалітети. Що стосується інформації про маршрути, GraphHopper 
покладається на відкриті джерела даних та створений користувачем 
контент, щоб надати точну та актуальну інформацію про маршрути та 
розклади громадського транспорту.\\


Переваги:
\begin{itemize}
    \item Можна використовувати в автономному режимі, що робить його 
    корисним у місцях з обмеженим інтернет-зв'язком.
    \item Надає детальну інформацію про маршрути та час у дорозі.
    \item Пропонує низку варіантів налаштування, включаючи можливість 
    додавання власних профілів маршрутів.
\end{itemize}

Недоліки:
\begin{itemize}
    \item Обмежене покриття, особливо в деяких регіонах за межами 
    Європи та Північної Америки.
    \item Може бути не таким зручним та інтуїтивно зрозумілим, як 
    деякі інші схожі додатки.
    \item Деякі користувачі повідомляють про неточності або помилки в 
    додатку, особливо щодо маршрутів або часу в дорозі.
    \item Обмежені функції та можливості порівняно з деякими іншими 
    схожими додатками.
\end{itemize}
