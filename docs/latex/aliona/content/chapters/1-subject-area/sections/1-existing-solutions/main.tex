\section{Загальний огляд програм для прокладаня транспорту}
\label{sec:existing-sulutions}

В останні роки, зі збільшенням щільності населення та заторів на дорогах у містах, пошук найкоротших та найшвидших маршрутів для громадського транспорту став більш важливим, ніж будь-коли. В Україні існує кілька програмних додатків, які пропонують рішення для пошуку транспортних маршрутів на основі розкладу, причому деякі з них навіть дозволяють невеликим компаніям додавати власні маршрути. Ось деякі з них:


\begin{itemize}
  \item Moovit --- популярний транспортний додаток, який в режимі реального часу надає інформацію про автобуси, трамваї та тролейбуси в більш ніж 100 містах по всьому світу. Додаток дозволяє користувачам планувати свої маршрути, знаходити найкращі варіанти пересування та відстежувати час прибуття транспорту[1].

  \item Easyway --- це транспортний додаток, який пропонує послуги, подібні до Moovit, але з фокусом на менші міста. Додаток дозволяє користувачам планувати свої маршрути, знаходити найкращі варіанти транспорту та відстежувати час прибуття транспорту. Easyway також дозволяє невеликим компаніям додавати власні маршрути в додаток, що робить його корисним інструментом як для перевізників, так і для пасажирів[2].

  \item GraphHopper ---  це механізм маршрутизації, який надає алгоритми та інструменти для планування, навігації та оптимізації маршрутів у логістиці та транспорті. Платформа з відкритим вихідним кодом пропонує настроювані рішення маршрутизації для різних видів транспорту[3].

  \item OpenTripPlanner --- це програмне забезпечення для планування поїздок з відкритим вихідним кодом, яке дозволяє користувачам знаходити найкращі маршрути та варіанти транспорту для своїх подорожей. Його можна налаштувати та інтегрувати з іншим програмним забезпеченням і додатками, щоб запропонувати безперебійні рішення для планування поїздок[4].

\end{itemize}

Нижче кожне з існуючих рішень буде розглянуто більш детально.

% Subsections
\subsection{Moovit}
\label{subsec:moovit-subsection}

Moovit - транспортний додаток, який в режимі реального часу інформує 
про автобусні, трамвайні та тролейбусні маршрути в більш ніж 100 
містах по всьому світу. Додаток завантажили понад 100 мільйонів 
користувачів і він доступний для пристроїв на iOS та Android.

Однією з важливих особливостей Moovit є його транспортна інформація в 
режимі реального часу. Додаток надає користувачам актуальну інформацію 
про розклади руху автобусів і трамваїв, а також в режимі реального 
часу відстежує місцезнаходження їх транспорту. Це дозволяє 
користувачам легко планувати свої поїздки і коригувати їх в разі 
затримок або скасувань.

Ще однією корисною функцією Moovit є функція планування поїздок. 
Додаток дозволяє користувачам вводити пункт відправлення та пункт 
призначення, а потім надає їм ряд варіантів транспорту, включаючи 
автобусні, трамвайні та тролейбусні маршрути. Користувачі можуть 
вибрати маршрут, який найкраще відповідає їхнім потребам, і додаток 
надасть їм покрокові вказівки, включно з пішохідними маршрутами до 
автобусних зупинок або трамвайних станцій та з них.

Moovit також пропонує ряд інших функцій, які полегшують подорожі 
користувачів. Наприклад, користувачі можуть налаштувати сповіщення для 
своїх регулярних автобусних або трамвайних маршрутів, щоб отримувати 
повідомлення про будь-які зміни в розкладі. У додатку також є функція 
"Живі вказівки", яка надає користувачам голосові інструкції щодо 
їхньої подорожі, що полегшує їхнє проходження, навіть якщо вони 
незнайомі з місцевістю.

На додаток до своєї основної функціональності, Moovit також пропонує 
ряд інших можливостей. Наприклад, користувачі можуть залишати відгуки 
та оцінки щодо свого транспортного досвіду, які допоможуть іншим 
користувачам обрати найкращі маршрути та варіанти перевезень. Додаток 
також має функцію під назвою "Moovit Insights", яка надає користувачам 
дані та статистику про використання транспорту в їхньому місті, наприклад, найбільш завантажені автобусні маршрути або середній час, 
який займає поїздка з одного району в інший.

В Україні Moovit доступний у вигляді мобільного додатку для платформ 
iOS і Android. Додаток надає користувачам в режимі реального часу 
інформацію про громадський транспорт, включаючи розклади, карти, 
графіки руху і плани маршрутів.  Компанія співпрацює з транспортними 
агентствами і збирає дані з їхніх розкладів і карт, які потім 
інтегруються в додаток.

В цілому, Moovit є корисним інструментом для всіх, кому потрібно 
орієнтуватися в громадському транспорті в місті. Інформація в режимі 
реального часу та функція планування поїздок дозволяють легко 
знаходити найкращі маршрути та варіанти пересування, а додаткові 
функції надають користувачам цінну інформацію про місцеву транспортну 
мережу.\\


Переваги:
\begin{itemize}
    \item Велика спільнота користувачів з активним користувацьким 
    контентом.
    \item Дані в реальному часі доступні для багатьох міст.
\end{itemize}

Недоліки:
\begin{itemize}
    \item Обмежена доступність у невеликих містах та сільській 
    місцевості.
    \item Обмежена точність даних у режимі реального часу.
    \item Доступний лише у вигляді мобільного додатку, без 
    веб-інтерфейсу.
\end{itemize}


\subsection{Moovit}
\label{subsec:moovit-subsection}


Переваги:
\begin{itemize}
    \item
\end{itemize}

Недоліки:
\begin{itemize}
    \item
\end{itemize}


\subsection{GraphHopper}
\label{subsec:graphhopper-subsection}

GraphHopper --- це потужний механізм маршрутизації, який пропонує 
гнучкі рішення для оптимізації маршрутів для різних сценаріїв 
використання. Програмне забезпечення можна використовувати для 
оптимізації маршрутів для автомобілів, велосипедів, пішоходів та 
громадського транспорту, що робить його універсальним рішенням для 
планування перевезень та оптимізації логістики.

Однією з ключових особливостей GraphHopper є його гнучкість. Програмне 
забезпечення можна легко налаштувати відповідно до конкретних потреб 
окремих підприємств чи організацій. Він пропонує цілий ряд варіантів 
конфігурації, які дозволяють користувачам налаштовувати алгоритми 
маршрутизації, вагові коефіцієнти та інші параметри відповідно до 
своїх потреб. Це робить його ідеальним рішенням для компаній, які 
потребують висококастомізованих рішень для маршрутизації, таких як 
компанії, що займаються доставкою або логістикою.

Ще однією сильною стороною GraphHopper є його масштабованість. 
Програмне забезпечення призначене для обробки великих обсягів даних і 
може легко обробляти запити на маршрутизацію для тисяч транспортних 
засобів або пунктів доставки. Це робить його чудовим вибором для 
компаній, які потребують оптимізації складних логістичних мереж або 
великих автопарків.

З точки зору користувацького досвіду, GraphHopper пропонує ряд API і 
веб-інтерфейсів, які дозволяють легко інтегрувати рішення для 
маршрутизації в існуюче програмне забезпечення або веб-сайти. 
Програмне забезпечення також добре задокументоване, з великою онлайн 
документацією та ресурсами підтримки, доступними для користувачів.

В Україні GraphHopper не так широко використовується, як деякі інші 
механізми маршрутизації, але він все ще є популярним вибором для 
підприємств та організацій, які потребують висококастомізованих рішень 
для маршрутизації. Зазвичай це програмне забезпечення використовується 
великими компаніями або організаціями з більш складними потребами в 
маршрутизації, такими як логістичні провайдери, транспортні компанії 
та муніципалітети. Що стосується інформації про маршрути, GraphHopper 
покладається на відкриті джерела даних та створений користувачем 
контент, щоб надати точну та актуальну інформацію про маршрути та 
розклади громадського транспорту.\\


Переваги:
\begin{itemize}
    \item Можна використовувати в автономному режимі, що робить його 
    корисним у місцях з обмеженим інтернет-зв'язком.
    \item Надає детальну інформацію про маршрути та час у дорозі.
    \item Пропонує низку варіантів налаштування, включаючи можливість 
    додавання власних профілів маршрутів.
\end{itemize}

Недоліки:
\begin{itemize}
    \item Обмежене покриття, особливо в деяких регіонах за межами 
    Європи та Північної Америки.
    \item Може бути не таким зручним та інтуїтивно зрозумілим, як 
    деякі інші схожі додатки.
    \item Деякі користувачі повідомляють про неточності або помилки в 
    додатку, особливо щодо маршрутів або часу в дорозі.
    \item Обмежені функції та можливості порівняно з деякими іншими 
    схожими додатками.
\end{itemize}


\subsection{OpenTripPlanner}
\label{subsec:otp-subsection}

OpenTripPlanner (OTP) --- це широко використовуваний планувальник 
маршрутів з відкритим вихідним кодом, який дозволяє користувачам 
знаходити найкращі маршрути громадського транспорту між двома 
пунктами. OTP був розроблений некомерційною організацією OpenPlans і 
зараз підтримується проектом OpenTripPlanner.

OTP призначений для роботи з широким спектром джерел даних, включаючи 
розклади та дані в реальному часі для громадського транспорту, 
дорожніх мереж та систем спільного користування велосипедами. Він 
використовує OpenStreetMap як основне джерело даних для карт і 
підтримує маршрутизацію для широкого спектру видів транспорту, 
включаючи пішохідний, велосипедний, автомобільний та громадський 
транспорт.

OTP дуже добре налаштовується і може бути адаптованим до потреб різних 
регіонів чи міст. Він пропонує гнучкий механізм маршрутизації, який 
дозволяє користувачам визначати власні уподобання, наприклад, уникати 
певних видів транспорту або надавати пріоритет певним видам транспорту 
над іншими.

OTP має зручний інтерфейс, який дозволяє користувачам легко вводити 
пункти відправлення та призначення, а також вказувати бажаний вид 
транспорту. Він надає детальну інформацію про кожен маршрут, включаючи 
приблизний час у дорозі, вартість проїзду та будь-які пересадки, які 
можуть знадобитися. Користувачі також можуть переглядати в режимі 
реального часу інформацію про стан громадського транспорту, включаючи 
затримки або скасування рейсів.

В Україні OpenTripPlanner не так широко використовується, як деякі 
інші інструменти планування поїздок, але він все ще є популярним 
вибором серед розробників і транспортних експертів, які потребують 
рішення для планування поїздок з високим ступенем налаштування. 
Система зазвичай розгортається місцевими органами влади або 
транспортними службами, які додають власні джерела даних і 
налаштовують параметри маршрутів відповідно до потреб своєї місцевості.\\


Переваги:
\begin{itemize}
    \item Надає детальну інформацію про маршрути та час у дорозі, 
    включаючи інформацію про транзит у реальному часі.
    \item Може використовуватися для планування мультимодальних 
    поїздок, включаючи комбінування велосипедних та транзитних 
    маршрутів.
    \item Пропонує ряд можливостей для налаштування транспортних даних.

\end{itemize}

Недоліки:
\begin{itemize}
    \item Обмежене покриття в деяких регіонах.
    \item Налаштування вимагає технічної експертизи.
    \item Потребує значних обчислювальних ресурсів.
\end{itemize}
