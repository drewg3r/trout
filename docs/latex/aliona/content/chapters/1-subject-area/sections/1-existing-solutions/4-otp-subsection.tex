\subsection{OpenTripPlanner}
\label{subsec:otp-subsection}

OpenTripPlanner (OTP) --- це широко використовуваний планувальник 
маршрутів з відкритим вихідним кодом, який дозволяє користувачам 
знаходити найкращі маршрути громадського транспорту між двома 
пунктами. OTP був розроблений некомерційною організацією OpenPlans і 
зараз підтримується проектом OpenTripPlanner.

OTP призначений для роботи з широким спектром джерел даних, включаючи 
розклади та дані в реальному часі для громадського транспорту, 
дорожніх мереж та систем спільного користування велосипедами. Він 
використовує OpenStreetMap як основне джерело даних для карт і 
підтримує маршрутизацію для широкого спектру видів транспорту, 
включаючи пішохідний, велосипедний, автомобільний та громадський 
транспорт.

OTP дуже добре налаштовується і може бути адаптованим до потреб різних 
регіонів чи міст. Він пропонує гнучкий механізм маршрутизації, який 
дозволяє користувачам визначати власні уподобання, наприклад, уникати 
певних видів транспорту або надавати пріоритет певним видам транспорту 
над іншими.

OTP має зручний інтерфейс, який дозволяє користувачам легко вводити 
пункти відправлення та призначення, а також вказувати бажаний вид 
транспорту. Він надає детальну інформацію про кожен маршрут, включаючи 
приблизний час у дорозі, вартість проїзду та будь-які пересадки, які 
можуть знадобитися. Користувачі також можуть переглядати в режимі 
реального часу інформацію про стан громадського транспорту, включаючи 
затримки або скасування рейсів.

В Україні OpenTripPlanner не так широко використовується, як деякі 
інші інструменти планування поїздок, але він все ще є популярним 
вибором серед розробників і транспортних експертів, які потребують 
рішення для планування поїздок з високим ступенем налаштування. 
Система зазвичай розгортається місцевими органами влади або 
транспортними службами, які додають власні джерела даних і 
налаштовують параметри маршрутів відповідно до потреб своєї місцевості.\\


Переваги:
\begin{itemize}
    \item Надає детальну інформацію про маршрути та час у дорозі, 
    включаючи інформацію про транзит у реальному часі.
    \item Може використовуватися для планування мультимодальних 
    поїздок, включаючи комбінування велосипедних та транзитних 
    маршрутів.
    \item Пропонує ряд можливостей для налаштування транспортних даних.

\end{itemize}

Недоліки:
\begin{itemize}
    \item Обмежене покриття в деяких регіонах.
    \item Налаштування вимагає технічної експертизи.
    \item Потребує значних обчислювальних ресурсів.
\end{itemize}
