\subsection{Moovit}
\label{subsec:moovit-subsection}

Moovit - транспортний додаток, який в режимі реального часу інформує 
про автобусні, трамвайні та тролейбусні маршрути в більш ніж 100 
містах по всьому світу. Додаток завантажили понад 100 мільйонів 
користувачів і він доступний для пристроїв на iOS та Android.

Однією з важливих особливостей Moovit є його транспортна інформація в 
режимі реального часу. Додаток надає користувачам актуальну інформацію 
про розклади руху автобусів і трамваїв, а також в режимі реального 
часу відстежує місцезнаходження їх транспорту. Це дозволяє 
користувачам легко планувати свої поїздки і коригувати їх в разі 
затримок або скасувань.

Ще однією корисною функцією Moovit є функція планування поїздок. 
Додаток дозволяє користувачам вводити пункт відправлення та пункт 
призначення, а потім надає їм ряд варіантів транспорту, включаючи 
автобусні, трамвайні та тролейбусні маршрути. Користувачі можуть 
вибрати маршрут, який найкраще відповідає їхнім потребам, і додаток 
надасть їм покрокові вказівки, включно з пішохідними маршрутами до 
автобусних зупинок або трамвайних станцій та з них.

Moovit також пропонує ряд інших функцій, які полегшують подорожі 
користувачів. Наприклад, користувачі можуть налаштувати сповіщення для 
своїх регулярних автобусних або трамвайних маршрутів, щоб отримувати 
повідомлення про будь-які зміни в розкладі. У додатку також є функція 
"Живі вказівки", яка надає користувачам голосові інструкції щодо 
їхньої подорожі, що полегшує їхнє проходження, навіть якщо вони 
незнайомі з місцевістю.

На додаток до своєї основної функціональності, Moovit також пропонує 
ряд інших можливостей. Наприклад, користувачі можуть залишати відгуки 
та оцінки щодо свого транспортного досвіду, які допоможуть іншим 
користувачам обрати найкращі маршрути та варіанти перевезень. Додаток 
також має функцію під назвою "Moovit Insights", яка надає користувачам 
дані та статистику про використання транспорту в їхньому місті, наприклад, найбільш завантажені автобусні маршрути або середній час, 
який займає поїздка з одного району в інший.

В Україні Moovit доступний у вигляді мобільного додатку для платформ 
iOS і Android. Додаток надає користувачам в режимі реального часу 
інформацію про громадський транспорт, включаючи розклади, карти, 
графіки руху і плани маршрутів.  Компанія співпрацює з транспортними 
агентствами і збирає дані з їхніх розкладів і карт, які потім 
інтегруються в додаток.

В цілому, Moovit є корисним інструментом для всіх, кому потрібно 
орієнтуватися в громадському транспорті в місті. Інформація в режимі 
реального часу та функція планування поїздок дозволяють легко 
знаходити найкращі маршрути та варіанти пересування, а додаткові 
функції надають користувачам цінну інформацію про місцеву транспортну 
мережу.\\


Переваги:
\begin{itemize}
    \item Велика спільнота користувачів з активним користувацьким 
    контентом.
    \item Дані в реальному часі доступні для багатьох міст.
\end{itemize}

Недоліки:
\begin{itemize}
    \item Обмежена доступність у невеликих містах та сільській 
    місцевості.
    \item Обмежена точність даних у режимі реального часу.
    \item Доступний лише у вигляді мобільного додатку, без 
    веб-інтерфейсу.
\end{itemize}
