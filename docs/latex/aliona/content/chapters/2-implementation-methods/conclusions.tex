\uchapter{Висновки до розділу 2}

У цьому розділі було зосереджено на найважливіших аспектах розробки веб-сервісу для планування маршрутів. Було розглянуто різні алгоритми, придатні для пошуку найкоротших маршрутів, зокрема алгоритм Дейкстри, алгоритм A*, алгоритм Беллмана-Форда, алгоритм Єна та алгоритм пошуку в ширину. Кожен алгоритм має свої сильні та слабкі сторони, і вибір залежить від таких факторів, як вимоги до продуктивності та характеристики графа.

Крім того, було враховано потребу включення інформації про розклад у планування маршрутів. Враховуючи розклади, сервіс може надавати користувачам оптимальні маршрути, які враховують час відправлення та прибуття, забезпечуючи ефективні та надійні подорожі.

Крім того, було заглиблено в розробку моделей даних для представлення основних сутностей системи, включаючи станції, маршрути, пересадки та пункти призначення. Ці моделі розроблені з урахуванням різних типів транспорту та конфігурацій маршрутів, що забезпечує гнучкість та масштабованість.

Для перетворення моделей у графове представлення було описано систематичний процес, що включає створення вузлів для станцій, ребер для сполучень і встановлення зв'язків між пунктами призначення та відповідними станціями. Таке графове представлення дозволяє проводити ефективні розрахунки маршрутів і полегшує включення алгоритмів для пошуку оптимальних маршрутів.

Завдяки вибору алгоритмів, включенню інформації про розклад та створенню відповідних моделей даних і графових представлень, веб-сервіс матиме необхідну основу для надання користувачам точних, ефективних і персоналізованих можливостей планування маршрутів. Ці компоненти працюють у взаємодії для забезпечення надійної та ефективної навігації, задовольняючи різноманітні потреби користувачів у різних транспортних сценаріях.