\uchapter{Висновки до розділу 1}

На основі аналізу існуючих сервісів планування маршрутів громадського 
транспорту можна зробити висновок, що зростає потреба в більш точних 
та ефективних інструментах, які надають інформацію про розклад руху, 
вартість проїзду та маршрути в режимі реального часу.

Багато користувачів сьогодні потребують простих у використанні та 
надійних додатків, які можуть швидко надати їм інформацію про найбільш 
зручні та доступні варіанти транспорту в їхньому районі. Існуючі 
рішення, такі як Moovit, Easyway, GraphHopper та OpenTripPlanner, 
пропонують різні функції та можливості для задоволення цих потреб.

Однак кожен з цих сервісів має свої обмеження та недоліки. Наприклад, 
деякі з них можуть не охоплювати всі регіони або види транспорту, тоді 
як інші можуть не надавати дані в режимі реального часу або мати 
обмежені алгоритми маршрутизації. Крім того, деякі сервіси можуть не 
дозволяти малим компаніям або місцевим перевізникам додавати власні 
маршрути, що може обмежувати доступність інформації.

Загалом, попит на послуги планування маршрутів громадського транспорту 
зростає, і є місце для нових та інноваційних рішень, які можуть більш 
ефективно задовольнити потреби користувачів. Усуваючи обмеження 
існуючих сервісів та надаючи зручний і персоналізований досвід, 
додаток, що розроблюється має потенціал стати успішним і задовольнити 
потреби багатьох користувачів громадського транспорту.