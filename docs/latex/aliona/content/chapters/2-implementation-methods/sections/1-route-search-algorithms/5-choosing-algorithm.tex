\subsection{Вибір алгоритму для вирішення задачі}
\label{subsec:bfs-subsection}


Як видно з розділів вище, існує багато алгоритмів для пошуку найкоротшого маршруту в графі. Для виконання поставленої задачі було обрано алгоритм Дейкстри. Він має кілька переваг, які роблять його гарним вибором для пошуку маршрутів у поставленій задачі.

Однією з головних переваг алгоритму Дейкстри є його простота та ефективність. Це простий алгоритм, який можна легко реалізувати на більшості мов програмування, і він має часову складність O(|E| + |V| log |V|), де |E| - кількість ребер, а |V| - кількість вершин у графі. Це робить його практичним вибором для великих графів з великою кількістю ребер і вершин. На відміну від деяких інших алгоритмів, таких як A* або Беллмана-Форда, він не вимагає евристичної функції або складної структури даних, як черга пріоритетів.

Ще однією перевагою алгоритму Дейкстри є те, що він гарантує знаходження найкоротшого шляху від початкової вершини до всіх інших вершин графа, якщо граф не містить ребер з від'ємною вагою, що для поставленої задачі є неможливим. Ця властивість важлива для багатьох застосувань, таких як маршрутизація в транспортних мережах, де знаходження найкоротшого шляху є важливим для оптимізації часу в дорозі та зменшення витрат.
 
Загалом, алгоритм Дейкстри є надійним та ефективним вибором для пошуку найкоротшого шляху в графі, що робить його популярним для таких додатків, як навігаційні системи та мережева маршрутизація. Простота, ефективність і гарантована оптимальність алгоритму Дейкстри роблять його чудовим вибором для пошуку маршрутів у розроблюваній системі.