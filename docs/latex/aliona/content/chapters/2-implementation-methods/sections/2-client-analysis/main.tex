\section{Аналіз потреб користувачів}
\label{sec:clients-analysis}

Потреби користувачів в функціях сервісу пошуку маршрутів громадського 
транспорту можуть відрізнятися залежно від контексту, в якому працює 
клієнт.

Для індивідуальних користувачів першочерговою потребою часто є зручний 
інтерфейс, який дозволяє їм швидко і легко знайти найкоротший та/або 
найшвидший маршрут з пункту А в пункт Б використовуючи громадський 
транспорт. Також для користувачів важливим є пошук маршрутів з 
найменьшою кількість пересадок та якнайменшим часом проведеним в 
транспорті. Їм також може знадобитися інформація в режимі реального 
часу про стан громадського транспорту, включаючи затримки, скасування 
та затори на дорогах.

Компанії чи організації, які надають послуги громадського транспорту, 
можуть потребувати програмного забезпечення, яке буде показувати 
користувачам всі послуги, які надаються цією компанією чи 
організацією. Тобто нові маршрути не загубляться серед тих з якими 
клієнти вже знайомі. Це безперечно є великим плюсом, тому що послуги 
не нав'язуються клієнтам. Їм також можуть знадобитись функції, які 
допоможуть оптимізувати маршрути та розклади, щоб скоротити час у 
дорозі та витрати, а також покращити якість обслуговування пасажирів.

Для невеликих компаній або організацій, які надають нішеві послуги 
громадського транспорту, такі як маршрутні автобуси або спеціалізовані 
маршрути, потреба може включати можливість додавати та керувати 
власними маршрутами в межах існуючих мереж громадського транспорту. 
Для цього може знадобитися інтеграція з існуючим програмним 
забезпеченням для пошуку маршрутів або розробка нового програмного 
забезпечення для задоволення їхніх конкретних потреб.

Загалом, ключові потреби в цій предметній області включають ефективні 
та точні алгоритми пошуку маршрутів, оновлення статусу громадського 
транспорту в режимі реального часу, зручні інтерфейси для 
індивідуальних користувачів, а також інструменти, що налаштовуються 
для компаній та організацій, які надають послуги громадського 
транспорту.

\subsection{Потреби індивідуальних користувачів}
\label{subsec:individual-users-needs}

Для більш дотального розуміння потреб індивідуальних користувачів слід 
розглянути групи людей, які користуються громадським транспортом:\\

	Люди, які їздять на роботу

    Люди, які їздять на роботу, є одними з основних користувачів громадського транспорту. Їм потрібна можливість швидко і легко знаходити найкоротші та найшвидші маршрути до місця призначення. Крім того, їм може знадобитися зберігати часто використовувані маршрути.\\

    Туристи

    Туристи є ще однією ключовою групою користувачів. Вони можуть бути не знайомі з системою громадського транспорту в новому місті і потребують можливості швидко і легко знаходити найкращі маршрути до популярних туристичних місць. Крім того, вони можуть захотіти зберегти свої маршрути для подальшого використання або поділитися ними з друзями та родиною.\\

    Студенти та школярі

    Студенти та школярі також значною мірою покладаються на громадський транспорт, щоб дістатися до школи чи університету. Їм потрібно вміти знаходити найшвидші та найкоротші маршрути до кампусу та додому. Їм також може знадобитися планувати свої маршрути з урахуванням розкладу занять або інших зобов'язань, таких як позакласні заходи або робота з неповним робочим днем.\\

	Ділові мандрівники

	Діловим мандрівникам часто потрібно дістатися до місця призначення якомога швидше. Вони повинні мати можливість знайти найшвидший і найефективніший маршрут, беручи до уваги трафік, розклад громадського транспорту та інші фактори.


Загалом, користувачі послуг маршрутизації громадського транспорту потребують точної та актуальної інформації про розклад та маршрути. Їм також потрібен зручний інтерфейс, який дозволяє швидко і легко знаходити потрібну інформацію. Іншими важливими функціями є можливість зберігати маршрути та ділитися ними, інформування в режимі реального часу про збої та затримки.
