\chapter{Постановка задачі}
\label{chap:implementation-methods}


Метою даної дипломної роботи є створення математичного та програмного
забезпечення для розпізнавання базових людських емоцій за фронтальним статичним
зображенням її обличчя.
	
До множини емоцій, що розпізнаватимуться, віднесено наступні:

\begin{enumerate}
	\item здивування;
	\item щастя;
	\item сум;
	\item відраза;
	\item злість.
\end{enumerate}

При розробленні відповідного забезпечення потрібно розв'язати наступні
завдання:

\begin{enumerate}
    \item проведення порівняльного аналізу існуючих методів розпізнавання
        емоційного стану людини (РЕСЛ) за зображенням обличчя;
    \item вибір та адаптація існуючого методу для вирішення задачі РЕСЛ;
    \item розробка програмного забезпечення на базі вибраного математичного
        методу;
    \item тестування розробленої автоматизованої системи.
\end{enumerate}

Реалізована система має задовольняти такі вимоги:
\begin{enumerate}
    \item мати високі показники ефективності розпізнавання;
    \item ураховувати нечіткість природи людських емоцій;
    \item бути спроможною навчатися на різних вибірках.
\end{enumerate}
