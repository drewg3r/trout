\intro


У наш час інформаційні технології досягли значного розвитку й
продовжують розвиватися. Разом із тим, підвищується актуальність задач
розпізнавання та класифікації образів, за розв'язання яких стає можливою
побудова машини, яка здатна сприймати світ за візуальною інформацією.

Однією з таких задач є задача розпізнавання машиною емоцій людини за виразом її
обличчя. Переживаючи емоції, люди невербально обмінюються інформацією, яку
можуть сприймати і ухвалювати на її основі певні рішення \cite{ekman} (про
зміну поведінки, теми розмови тощо). Інформаційна система, що сприймає людські
емоції, могла б забезпечити ліпший інтерфейс взаємодії з людиною, адаптуватися
під її настрій. Емоції могли б стати невербальним фільтром під час пошуку
фільмів, бути причиною зміни ігрового процесу чи братись до уваги під час
оцінки істинності вербальної інформації з боку системи.

На сьогоднішній день розроблено низку комерційних програмних систем, які
розпізнають емоції за зображенням обличчя. Проте, вони мають певні обмеження
або значну похибку розпізнавання, тому актуальною є проблема ідентифікації
емоцій.

У даній роботі проаналізовано доступні комерційні та некомерційні математичні
методи та програмні рішення для розв'язання задачі ідентифікацї емоцій людини
за зображенням її обличчя з метою виділення та реалізації технології, що
враховувала би природну нечіткість параметрів, які визначають емоції. Оскільки
одна емоція має множину виявів у виразах обличчя, а кожному виразу обличчя може
відповідати не одна емоція зі своїм ступенем інтенсивності, доцільним є
використання підходів, що базуються на математичному апараті нечіткої логіки.
